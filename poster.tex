\documentclass[a0paper,25pt,portrait,blockverticalspace=20mm, colspace=20mm]{tikzposter}

\usepackage{etex}
%\usepackage{authblk}
\usepackage{amsmath,amsthm,mathrsfs,amssymb,amsfonts,dsfont,nicefrac,stmaryrd,yhmath}
\usepackage{mathtools,cancel}
\mathtoolsset{showonlyrefs=true}
\usepackage{multicol}
\usepackage{csquotes}
\usepackage{bold-extra,moresize}
\usepackage{layout}
\usepackage{tikz,pgfplots}
\pgfplotsset{compat=newest}
\usepackage{pgfmath,pgffor}
\usetikzlibrary{plotmarks}
\usepackage{pgfplotstable}
\usepackage{pgfpages}
\usepackage{tikz-3dplot}

\usetikzlibrary{datavisualization, patterns}
\usetikzlibrary{arrows,positioning,shapes,trees,mindmap,shadows} 
\usepgfplotslibrary{fillbetween}
\usepgfplotslibrary{colormaps}
\usepgfplotslibrary{colorbrewer}
\usepgfplotslibrary{external}
\usepgfplotslibrary{groupplots}
\usetikzlibrary{decorations.markings,intersections,calc}
\usetikzlibrary{decorations.pathreplacing}
\usetikzlibrary{fit}
%%%%%%%%%%%%%
%
% Defining commands
%
%%%%%%%%%%%%%
\newcommand{\calF}{\mathcal{F}}
\newcommand{\calE}{\mathcal{E}}
\newcommand{\calT}{\mathcal{T}}
\newcommand{\calS}{\mathcal{S}}
\newcommand{\calR}{\mathcal{R}}

\newcommand{\frakR}{\mathfrak{R}}

\newcommand{\grad}{\nabla}

\newcommand{\R}{\mathbb{R}}
\newcommand{\C}{\mathbb{C}}
\newcommand{\Q}{\mathbb{Q}}
\newcommand{\N}{\mathbb{N}}
\newcommand{\Z}{\mathbb{Z}}
\newcommand{\K}{\mathbb{K}}
\newcommand{\Y}{\mathbb{Y}}

\renewcommand{\H}{\mathbb{H}}
\newcommand{\V}{\mathbb{V}}

\newcommand{\dis}{\displaystyle}
\newcommand{\abs}[1]{\left|#1\right|}
\newcommand{\eps}{\varepsilon}
\newcommand{\norme}[1]{\left\|#1\right\|}
\newcommand{\norm}[1]{\left\|#1\right\|}

\renewcommand{\leq}{\leqslant}
\renewcommand{\geq}{\geqslant}
\renewcommand{\tilde}{\widetilde}
\newcommand{\scalaire}[2]{\left<#1\,,#2\right>}
\newcommand{\scalar}[2]{\left<#1\,,#2\right>}


%
%
%%%%%%%%%%%%
%
% Redefining style
%
%%%%%%%%%%%%%%
\tikzposterlatexaffectionproofoff
%\usetheme{Wave}

\definecolor{myred}{rgb}{0.5,0,0.1}
\definecolor{mygreen}{rgb}{0,0.5,0.1}
\definecolor{mybrown}{rgb}{0.75,0.5,0}
\definecolor{mygray}{rgb}{0.5,0.5,0.5}
 \definecolorstyle{myColorStyle}{%
\colorlet{colorOne}{mygreen}
 \colorlet{colorTwo}{mygray} 
 \colorlet{colorThree}{mybrown}
 %
}{
% Background Colors 
\colorlet{backgroundcolor}{colorOne!50} 
\colorlet{framecolor}{colorTwo!50}
% Title Colors 
\colorlet{titlefgcolor}{black} 
\colorlet{titlebgcolor}{colorOne!60}
% Block Colors 
\colorlet{blocktitlebgcolor}{colorThree!60} 
\colorlet{blocktitlefgcolor}{black} 
\colorlet{blockbodybgcolor}{colorTwo!5} 
\colorlet{blockbodyfgcolor}{black}
% Innerblock Colors
\colorlet{innerblocktitlebgcolor}{colorOne!60} 
\colorlet{innerblocktitlefgcolor}{black} 
\colorlet{innerblockbodybgcolor}{colorOne!20} 
\colorlet{innerblockbodyfgcolor}{black}
% Note colors 
\colorlet{notefgcolor}{black} 
\colorlet{notebgcolor}{yellow!50} 
\colorlet{notefrcolor}{white}
}

\usecolorstyle{myColorStyle}
%\usecolorstyle[colorPalette=GreenGrayViolet]{Russia}
\usebackgroundstyle{BottomVerticalGradation}
\usenotestyle{Corner}

\usetitlestyle{Wave}
\useblockstyle{Envelope}

\newlength{\plotwidth}
\newlength{\plotheight}

\setlength{\plotheight}{15cm}
\setlength{\plotwidth}{20cm}

\newcommand{\todoVD}[2][]%
{\todo[color=green!25,#1]{\footnotesize{\bf Vincent:} #2}}

\newcommand{\todoJA}[2][]%
{\todo[color=cyan!25,#1]{\footnotesize{\bf Julen:} #2}}

\newcommand{\todoDP}[2][]%
{\todo[color=blue!25,#1]{\footnotesize{\bf David:} #2}}

\newcommand{\todoFC}[2][]%
{\todo[color=red!25,#1]{\footnotesize{\bf Felipe:} #2}}


\title{\parbox{\linewidth}{\centering A painless multi-level automatic goal-oriented $hp$-adaptive coarsening strategy for elliptic and non-elliptic problems}}
\titlegraphic{\includegraphics[height=4.5cm]{Figures/logo_bcam} \quad \includegraphics[height=4.5cm]{Figures/UPV-EHU-logo}}

\author{{\underline{Vincent Darrigrand}, Julen Alvarez-Aramberri, Felipe V. Caro, Elisabete Alberdi, David Pardo}}
%\institute{MathMode}
\settitle{ \centering \vbox{
     \centering
     \color{titlefgcolor} {\bfseries \Huge \@title \par}
     \vspace*{1em}
     {\LARGE \it \@author \par} \vspace*{1em}  \@titlegraphic \\[\TP@titlegraphictotitledistance]\vspace*{-1em} {\Large \@institute}
}}

% !TEX root = poster.tex
\tikzset{mark size=3}

\newcommand{\PbName}{}
\newcommand{\ColName}{}
\newcommand{\FigurePath}{Figures}
\newcommand{\DataPath}{}

%==========
% \errorplotlog
%==========
\newcommand{\errorplotlog}[3][]{

  \renewcommand{\DataPath}{\FigurePath/#2}
  \renewcommand{\PbName}{#2}
  \renewcommand{\ColName}{#3} 
  
  %\findmax{\DataPath/h/order_2/outputs.txt}{nr_dof}{\maxdof}
  %\findmin{\DataPath/h/order_2/outputs.txt}{nr_dof}{\mindof}
  %\findmaxDbl{\DataPath/hp/order_1/outputs.txt}{Error}{\maxerror}
  %\findminDbl{\DataPath/hp/order_1/outputs.txt}{Error}{\minerror}

  \tikzset{/tikz/external/export next=false}
  \begin{tikzpicture}
    \pgfplotsset{xmode=log}

    \begin{axis}[width=\plotwidth,height=\plotheight,
		xlabel={Number of DoFs},
		%x dir=reverse,
		ylabel=Relative error in \% (log scale),
		ymode=log,
		xmode=log,
		%ymin=\minerror,
		%ymax=\maxerror,
		%xmin=\mindof,
		%xmax=\maxdof,
		ytick={0.0000001, 0.0001, 0.1, 100},
		%extra y ticks={\minerror,\maxerror},
		%yticklabel pos=right,
		ylabel near ticks,
		xlabel near ticks,
		enlargelimits=true,
		legend style={draw=black,fill=white,legend cell align=left, at={(0.5,1.01)}, anchor=south},
		legend columns=-1
		]
		\addplot+[ line width=1pt] table[x expr=\thisrow{nr_dof},y expr=\thisrow{#3}] {\FigurePath/\PbName/hp/order_1/outputs.txt}node[pos=0.9, pin={[pin edge=solid]180:$hp$}]{};
		
		\addplot+[ line width=1pt] table[x expr=\thisrow{nr_dof},y expr=\thisrow{#3}] {\FigurePath/\PbName/h/order_1/outputs.txt}node[pos=0.9, pin={[pin edge=solid]90:$h$ ($p=1$)}]{};
		
		\addplot+[ line width=1pt] table[x expr=\thisrow{nr_dof},y expr=\thisrow{#3}] {\FigurePath/\PbName/h/order_2/outputs.txt} node[pos=0.9, pin={[pin edge=solid]-90:$h$ ($p=2$)}]{};
    \end{axis}
  \end{tikzpicture}
} 
%======
% \plothp
%======
\newcommand{\plothp}[2][]{

  \renewcommand{\DataPath}{\FigurePath/#2/hp/order_1}%\PbName/hp
  \findmax{\DataPath/outputs.txt}{iter}{\endIter}
  
   \FormatIntegerThreeDigit{\endIter}{\endIterformated}
   
    %\centering
    \begin{minipage}[b]{0.45\linewidth}
    \raggedleft
    %\begin{tikzfigure}[Final $hp$-adapted mesh with polynomial orders in the $x$-direction.]
    \plothpmesh{#2/hp/order_1}{\endIterformated}{X}
    %\end{tikzfigure}%
    \end{minipage} \quad
    \begin{minipage}[b]{0.45\linewidth}
    \raggedright
    %\begin{tikzfigure}[Final $hp$-adapted mesh with polynomial orders in the $y$-direction.]
     \plothpmesh{#2/hp/order_1}{\endIterformated}{Y}
    %\end{tikzfigure}%
    \end{minipage}% 
}

\newcommand{\plothpmesh}[3]{
  \renewcommand{\DataPath}{\FigurePath/#1}

  \lastelement{\DataPath/outputs.txt}{MaxP}{\maxp}
  \lastelement{\DataPath/outputs.txt}{MinP}{\minp}

  \pgfplotsset{colormap/YlOrRd} 
  \tikzset{/tikz/external/export next=false}
  \begin{tikzpicture}
    \begin{axis}[name=master,
        width=0.95\textwidth,height=0.95\textwidth,
        enlargelimits=false,
        xlabel=$x$,
        ylabel=$y$,
        xlabel near ticks,
        ylabel near ticks,
        xtick={0,0.5,1},
        ytick={0,0.5,1},
        colorbar horizontal,
        colorbar style={%
            xtick={1,2,...,\maxp},
            title={Order of approximation},
            at={(master.above north west)},anchor=south west,
            yshift=-5pt,
            title style={
                yshift=20pt
              },
            yshift=0.1em,
            xticklabel pos=upper,
          },
        %point meta min=\minp,
        point meta min=1,
        point meta max=\maxp,
      ]
      %
      \addplot [forget plot] graphics[xmin=0,xmax=1,ymin=0,ymax=1] {\DataPath/mesh_#3_#2.png};

    \end{axis}
  \end{tikzpicture}
}
%===============
% elemental routines
%===============
\newcommand{\FormatIntegerTwoDigit}[2]{
\pgfmathsetbasenumberlength{2}
\pgfmathbasetodec#2{#1}{10}
}

\newcommand{\FormatIntegerThreeDigit}[2]{
\pgfmathsetbasenumberlength{3}
\pgfmathbasetodec#2{#1}{10}
}

\newcommand{\getelem}[4]{
    \pgfplotstablegetelem{#3}{#2}\of{#1}%
    \pgfmathtruncatemacro#4{\pgfplotsretval}
    \pgfplotstableclear{\datatable}
}


\newcommand{\getelemdbl}[4]{
    \pgfplotstablegetelem{#3}{#2}\of{#1}%
    \pgfmathsetmacro#4{\pgfplotsretval}    
    \pgfplotstableclear{\datatable}
}



\newcommand{\findmax}[3]{
  \pgfplotstableread{#1}{\datatable}
  \pgfplotstablesort[sort key={#2},sort cmp={float >}]{\sorted}{\datatable}%
  \pgfplotstablegetelem{0}{#2}\of{\sorted}%
  \pgfmathtruncatemacro#3{\pgfplotsretval}
  \pgfplotstableclear{\datatable}
}

\newcommand{\findmin}[3]{
  \pgfplotstableread{#1}{\datatable}
  \pgfplotstablesort[sort key={#2},sort cmp={float <}]{\sorted}{\datatable}%
  \pgfplotstablegetelem{0}{#2}\of{\sorted}%
  \pgfmathtruncatemacro#3{\pgfplotsretval}
  \pgfplotstableclear{\datatable}
  %
}

\newcommand{\findmaxDbl}[3]{
  \pgfplotstableread{#1}{\datatable}
  \pgfplotstablesort[sort key={#2},sort cmp={float >}]{\sorted}{\datatable}%
  \pgfplotstablegetelem{0}{#2}\of{\sorted}%
  \pgfmathsetmacro#3{\pgfplotsretval}
  \pgfplotstableclear{\datatable}
  %
}

\newcommand{\lastelement}[3]{
\pgfplotstableread{#1}{\mytable}
\pgfplotstablegetrowsof{\mytable} %Determine no. of rows
\pgfmathtruncatemacro{\rows}{\pgfplotsretval}
\pgfmathtruncatemacro{\lastrow}{\rows-1}
\pgfplotstablegetelem{\lastrow}{#2}\of{\mytable}
\pgfmathsetmacro#3{\pgfplotsretval}
}


\newcommand{\findminDbl}[3]{
  \pgfplotstableread{#1}{\datatable}
  \pgfplotstablesort[sort key={#2},sort cmp={float <}]{\sorted}{\datatable}%
  \pgfplotstablegetelem{0}{#2}\of{\sorted}%
  \pgfmathsetmacro#3{\pgfplotsretval}
  \let#3=\pgfplotsretval
  \pgfplotstableclear{\datatable}
  %
}

\pgfplotsset{
  log x ticks with fixed point/.style={
      xticklabel={
        \pgfkeys{/pgf/fpu=true}
        \pgfmathparse{exp(\tick)}%
        \pgfmathprintnumber[fixed relative, precision=3]{\pgfmathresult}
        \pgfkeys{/pgf/fpu=false}
      }
  },
  log y ticks with fixed point/.style={
      yticklabel={
        \pgfkeys{/pgf/fpu=true}
        \pgfmathparse{exp(\tick)}%
        \pgfmathprintnumber[fixed relative, precision=3]{\pgfmathresult}
        \pgfkeys{/pgf/fpu=false}
      }
  }
}

\pgfplotscreateplotcyclelist{custom list style}{%
color=red, solid, every mark/.append style={solid}, mark=*\\%
color=blue, loosely dotted, every mark/.append style={solid}, mark=triangle*\\%
color=black!50!green,dotted, every mark/.append style={solid}, mark=square*\\%
color=black, dashed, every mark/.append style={solid},mark=diamond*\\%
color=black!50!red, densely dotted, every mark/.append style={solid}, mark=otimes*\\%
color=black!50!yellow, loosely dashed, every mark/.append style={solid},mark=*\\%
color=black!50!blue,densely dashed, every mark/.append style={solid},mark=square*\\%
color=black!50!magenta,dashdotted, every mark/.append style={solid},mark=otimes*\\%
color=black!50!orange,dashdotdotted, every mark/.append style={solid},mark=star\\%
color=black!50!cyan, densely dashdotted,every mark/.append style={solid},mark=diamond*\\%
}

\pgfplotsset{%
%    every axis plot/.append style= {line width=2pt},
        every axis plot/.append style= ultra thick,
    cycle list name=custom list style,
%    log x ticks with fixed point,
%    legend style={at={(0.5,1.03)},anchor=south},
%    legend columns=3,
%    legend pos= outer north east,
%    legend cell align=left,
    }
 
\begin{document} 
  
  \maketitle
  \begin{columns}%%%%%%%%%%%%%%%%%%%%%%%%%%%%%%%%%%%%%%%%%%%%%%%%%%%%%%%%%%%%%%%%%%%%%%%%%%
  %
 \column{0.5} 
 %
 \block{Data structures}{ %
%To ensure continuity and avoid the so-called \emph{hanging nodes} by construction, we use the data structures proposed by Zander et al. (2015), which impose a massive number of Dirichlet nodes throughout the overlay mesh(es). 
%Our \enquote{painless} approach operates always on a single mesh. 
%To do so, it requires defining a projector that simulates the presence of a coarse mesh without the trouble of handling one.
%For a given subset of basis functions $\calS\subset \calF$ that generates the space $\H_\calS \subset \H_\calF$
%Let $\calF = \{ \phi_{i} \}_{i=1}^{n_{\calF}}$ be a set of basis functions $\phi_i$, $\H_{\calF} \coloneqq \textrm{span}\{\phi_{1}, \dots, \phi_{n_{\calF}}\}$, and $n_{\calF} = \textrm{dim}(\H_{\calF})$.
%that is, we extract the coefficients of $u_\calF$ corresponding to a given subset of basis functions $\calS\subset \calF$, and we set the others to zero.
\innerblock{}{We define our \emph{projection operator} $\Pi_{\calF}^{\calS}$: $\H _{\calF} \longrightarrow \H_{\calS}$ as
\begin{equation}
 \Pi_{\calF}^{\calS} u_{\calF} \coloneqq \sum_{\phi_{i} \in \calS} u_{i} \phi_{i},
\end{equation}
for a given subset of basis functions $\calS\subset \calF$.}
%
%For any element $K$, we denote by $\calR_K$ the set of \emph{removable} basis functions associated to $K$, by $\abs{\calR_K}$ its cardinality, and by $\H_{\calR_K}$ its associated space.
We express any $u_\calF \in \H_\calF$, as
%
\begin{equation}
  u_{\calF} =  \Pi_{\calF}^{\calE_K} u_{\calF} + \Pi_{\calF}^{\calR_K} u_{\calF},
\end{equation}
where $\calE_K \coloneqq \calF \setminus \calR_K$ is the subset of \emph{essential} basis functions and $\calR_K$ the set of \emph{removable} basis functions associated to $K$.
%Since we consider a single mesh at a time, the solution $u_{\calE_K}$ in $\calE_K$ is never computed. Instead, we employ the projection of $u_\calF$ into $\calE_K$ to approximate it when necessary. 
We illustrate of a 1D multi-level $hp$-grid with hierarchical basis functions.
\innerblock{}{
\begin{center}
	\begin{tikzpicture}[scale=3,x=4cm,y=2cm,decoration={markings,%switch on markings mark=% actually add a mark
     	mark=at position 0 with{\draw (0pt,-2pt) -- (0pt,2pt);},
     	mark=at position 1 with{\draw (0pt,-2pt) -- (0pt,2pt);},
	}
	]
	\node(origin1) at (0,1) {};
	\node(origin2) at (1,1) {};
	\node(origin3) at (2,1) {};

	\node(father1) at (0,0) {};
	\node(father2) at (0.25,0) {};
	\node(father3) at (0.5,0) {};
\node(father4) at (0.75,0) {};
\node(father5) at (1,0) {};
\node(father6) at (1.25,0) {};
\node(father7) at (1.5,0) {};
\node(father8) at (1.75,0) {};
\node(father9) at (2,0) {};
\node(father10) at (2.25,0) {};
\node(father11) at (2.5,0) {};
\node(father12) at (2.75,0) {};
\node(father13) at (3,0) {};

\node(son1) at (0,-1) {};
\node(son1a) at (0.125,-1) {};
\node(son1b) at (0.25,-1) {};
\node(son1c) at (0.375,-1) {};
\node(son2)  at (0.5,-1) {};
\node(son3)  at (0.75,-1) {};
\node(son4)  at (1,-1) {};
\node(son10)  at (2,-1) {};
\node(son11) at (2.25,-1) {};
\node(son12) at (2.5,-1) {};
\node(son13) at (2.75,-1) {};
\node(son14) at (3,-1) {};

\node(grandson1) at (0,-2) {};
\node(grandson2)  at (0.25,-2) {};
\node(grandson3)  at (0.5,-2) {};
\node(grandson4)  at (0.625,-2) {};
\node(grandson5)  at (0.75,-2) {};
\node(grandson6)  at (0.875,-2) {};
\node(grandson7)  at (1,-2) {};
\node(grandson8)  at (1.25,-2) {};

\node(ggson1) at (0,-3) {};
\node(ggson2)  at (0.25,-3) {};
\node(ggson3)  at (0.5,-3) {};
\node(ggson4)  at (0.625,-3) {};
\node(ggson5)  at (0.75,-3) {};
\node(ggson6)  at (0.875,-3) {};
\node(ggson7)  at (1,-3) {};
\node(ggson8)  at (1.25,-3) {};


\node(basis0_0) at ($(father1)+(0,0.5)$){};
\node(basis0_1) at ($(father5)+(0,0.5)$){};
\node(basis0_2) at ($(father9)+(0,0.5)$){};
\node(basis0_3) at ($(father13)+(0,0.5)$){};
\node(basis1_1) at ($(son2)+(0,0.5)$){};
\node(basis1_2) at ($(son11)+(0,0.5)$){};
\node(basis1_3) at ($(son12)+(0,0.5)$){};
\node(basis1_4) at ($(son13)+(0,0.5)$){};
\node(basis2_1) at ($(grandson5)+(0,0.5)$){};
\node(basis2_2) at ($(grandson8)+(0,0.5)$){};
\node(basis2_3) at ($(grandson4)+(0,0.25)$){};
\node(basis2_4) at ($(grandson6)+(0,0.25)$){};
\node(basis3_1) at ($(ggson5)+(0,0.5)$){};
\node(basis3_2) at ($(ggson8)+(0,0.5)$){};
\node(basis3_3) at ($(ggson4)+(0,0.5)$){};
\node(basis3_4) at ($(ggson6)+(0,0.5)$){};


%%%%%% HORIZONTAL LINES
% level 1
\draw [postaction={decorate}] (father1.center) -- (father5.center) node[pos=0.5](elemfather1){};
\draw [postaction={decorate}] (father5.center) -- (father9.center) node[pos=0.5](elemfather2){};
\draw [postaction={decorate}] (father9.center) -- (father13.center) node[pos=0.5](elemfather3){};

% level 2

\draw [postaction={decorate,red}] (son1.center) -- (son2.center) node[pos=0.5](elemson1){};
\draw [postaction={decorate}] (son2.center) -- (son4.center) node[pos=0.5](elemson2){};

\draw [postaction={decorate,red}] (son10.center) -- (son12.center) node[pos=0.5](elemson31){};
\draw [postaction={decorate}] (son12.center) -- (son14.center) node[pos=0.5](elemson4){};

% level 2
\draw [postaction={decorate}] (grandson3.center) -- (grandson5.center) node[pos=0.5](elemgrandson1){};
\draw [postaction={decorate}] (grandson5.center) -- (grandson7.center) node[pos=0.5](elemgrandson2){};

% k3
%\draw [postaction={decorate}] (ggson3.center) -- (ggson5.center) node[pos=0.5](elemggson1){};
%\draw [postaction={decorate}] (ggson5.center) -- (ggson7.center) node[pos=0.5](elemggson2){};

%%%%%%%%% BOTTOM LINE %%%%%%%%%

\node(gggson1) at (0,-2.5) {};
\node(gggson2)  at (0.125,-2.5) {};
\node(gggson3)  at (0.25,-2.5) {};
\node(gggson4)  at (0.375,-2.5) {};
\node(gggson5)  at (0.5,-2.5) {};
\node(gggson6)  at (0.625,-2.5) {};
\node(gggson7)  at (0.75,-2.5) {};
\node(gggson8a)  at (0.8125,-2.5) {};
\node(gggson8)  at (0.875,-2.5) {};
\node(gggson8b)  at (0.9325,-2.5) {};
\node(gggson9)  at (1,-2.5) {};
\node(gggson10)  at (1.125,-2.5) {};
\node(gggson11)  at (1.25,-2.5) {};
\node(gggson12)  at (1.375,-2.5) {};
\node(gggson13)  at (1.5,-2.5) {};
\node(gggson14)  at (1.625,-2.5) {};
\node(gggson15)  at (1.75,-2.5) {};
\node(gggson16)  at (1.875,-2.5) {};
\node(gggson17)  at (2,-2.5) {};
\node(gggson18)  at (2.125,-2.5) {};
\node(gggson19)  at (2.25,-2.5) {};
\node(gggson20)  at (2.375,-2.5) {};
\node(gggson21)  at (2.5,-2.5) {};
\node(gggson22)  at (2.625,-2.5) {};
\node(gggson23)  at (2.75,-2.5) {};
\node(gggson24)  at (2.875,-2.5) {};
\node(gggson25)  at (3,-2.5) {};

\draw [|-] (gggson1.center) -- (gggson5.center) node[pos=0.5](){};
\draw [|-] (gggson5.center) -- (gggson7.center) node[pos=0.5](){};
\draw [|-] (gggson7.center) -- (gggson9.center) node[pos=0.5](){};
\draw [|-] (gggson9.center) -- (gggson17.center) node[pos=0.5](){};
\draw [|-] (gggson17.center) -- (gggson21.center) node[pos=0.5](){};
\draw [|-|] (gggson21.center) -- (gggson25.center) node[pos=0.5](){};


%%%%%%%%% VERTICAL LINES %%%%%%%%%

% From level 0
\draw[dotted] (father1.center) -- (gggson1.center);
\draw[dotted] (father5.center) -- (gggson9.center);
\draw[dotted] (father9.center) -- (gggson17.center);
\draw[dotted] (father13.center) -- (gggson25.center);

%\draw[loosely dotted,very thin] (father6.center) -- (gggson11.center);
%\draw[loosely dotted,very thin] (father7.center) -- (gggson13.center);
%\draw[loosely dotted,very thin] (father8.center) -- (gggson15.center);
%\draw[loosely dotted,very thin] (father11.center) -- (gggson21.center);

% From level 1
\draw[loosely dotted] (son2.center) -- (gggson5.center);
%\draw[loosely dotted] (son4.center) -- (gggson9.center);

%\draw[loosely dotted, very thin] (son1a.center) -- (gggson2.center);
%\draw[loosely dotted, very thin] (son1b.center) -- (gggson3.center);
%\draw[loosely dotted, very thin] (son1c.center) -- (gggson4.center);
%\draw[loosely dotted, very thin] (son11.center) -- (gggson19.center);
\draw[loosely dotted, very thin] (son12.center) -- (gggson21.center);
%\draw[loosely dotted, very thin] (son13.center) -- (gggson23.center);


% From level 2
\draw[loosely dotted] (grandson5.center) -- (gggson7.center);

%\draw[loosely dotted, very thin] (ggson4.center) -- (gggson6.center);
%\draw[loosely dotted, very thin] (ggson6.center) -- (gggson6.center);
%\draw[loosely dotted, very thin] (ggson6.center) -- (gggson8.center);

%\node at (gggson1)  {\Cross};
%\node at (gggson2)  {\Cross};
%\node at (gggson3)  {\Cross};
%\node at (gggson4)  {\Cross};
%\node at (gggson5)  {\Cross};
%\node at (gggson6)  {\Cross};
%\node at (gggson7)  {\Cross};
%\node at (gggson8a)  {\Cross};
%\node at (gggson8)  {\Cross};
%\node at (gggson8b)  {\Cross};
%\node at (gggson9)  {\Cross};
%\node at (gggson11)  {\Cross};
%\node at (gggson13)  {\Cross};
%\node at (gggson15)  {\Cross};
%\node at (gggson17)  {\Cross};
%%\node at (gggson19)  {\Cross};
%\node at (gggson21)  {\Cross};
%%\node at (gggson23)  {\Cross};
%\node at (gggson25)  {\Cross};

%%%%%%%%% BASIS FUNCTIONS %%%%%%%%%
%%%%%% level 0

% linear basis function

\draw[color=black, thick] (basis0_0.center) -- (father5.center);
\draw[color=black, thick] (father1.center) -- (basis0_1.center) -- (father9.center);
\draw[color=black, thick] (father5.center) -- (basis0_2.center) -- (father13.center);
\draw[color=black, thick] (basis0_3.center) -- (father9.center);

% degree 2 basis function
\draw[color=black, thick]    (father5.center) .. controls ($(elemfather2)+(-0.05,+0.15)$) and ($(elemfather2)+(0.05,0.15)$) ..  (father9.center) ;
%\draw[color=black, thick]    (father9.center) .. controls ($(elemfather3)+(-0.05,+0.15)$) and ($(elemfather3)+(0.05,0.15)$) ..  (father13.center) ;

% degree 3 basis function
\draw[color=red, ultra thick]   (father5.center) .. controls ($(elemfather2)+(-0.01,-0.25)$) and ($(elemfather2)+(0.00,0.25)$) ..  (father9.center);

%%%%% level 1
% linear basis function
\draw[color=black, thick] (son1.center) -- (basis1_1.center) -- (son4.center);
\draw[color=red, ultra thick] (son10.center) -- (basis1_3.center) -- (son14.center);
%\draw[color=red, thick] (son10.center) -- (basis1_2.center) -- (son12.center);
%\draw[color=red, thick] (son12.center) -- (basis1_4.center) -- (son14.center);

% degree 2 basis function
\draw[color=black, thick]    (son1.center) .. controls ($(elemson1)+(-0.05,+0.15)$) and ($(elemson1)+(0.05,0.15)$) ..  (son2.center);

% degree 3 basis function
\draw[color=red, ultra thick]   (son1.center) .. controls ($(elemson1)+(-0.0,-0.25)$) and ($(elemson1)+(0.0,0.25)$) ..  (son2.center);


%%%%% level 2

% linear basis function
\draw[color=black, thick] (grandson3.center) -- (basis2_1.center) -- (grandson7.center);
%\draw[color=red, thick]    (grandson3.center) -- (basis2_3.center) -- (grandson5.center);
%\draw[color=red, thick]    (grandson5.center) -- (basis2_4.center) -- (grandson7.center);

% degree 2 basis function
\draw[color=red, ultra thick]    (grandson3.center) .. controls ($(elemgrandson1)+(-0.05,+0.15)$) and ($(elemgrandson1)+(0.05,0.15)$) ..  (grandson5.center);
\draw[color=black, thick]    (grandson5.center) .. controls ($(elemgrandson2)+(-0.05,+0.15)$) and ($(elemgrandson2)+(0.05,0.15)$) ..  (grandson7.center);

% degree 3 basis function
\draw[color=red, ultra thick]   (grandson5.center) .. controls ($(elemgrandson2)+(-0.0,-0.25)$) and ($(elemgrandson2)+(0.0,0.25)$) ..  (grandson7.center);


%%%%% k3

% linear basis function
%\draw[color=red, thick]    (ggson3.center) -- (basis3_3.center) -- (ggson5.center);
%\draw[color=red, thick]    (ggson5.center) -- (basis3_4.center) -- (ggson7.center);

%%%%%%%%% CIRCLES %%%%%%%%%

%%%%%% level 0

\node[circle,draw, fill=black, inner sep=0pt,minimum size=5pt] at ($(father1)+(0,0)$){};
%\node[circle,draw, fill=white, inner sep=0pt,minimum size=5pt] at ($(father1a)+(0,0)$){};
%\node[circle,draw, fill=white, inner sep=0pt,minimum size=5pt] at ($(father1b)+(0,0)$){};
\node[circle,draw, fill=black, inner sep=0pt,minimum size=5pt] at ($(father5)+(0,0)$){};
%\node[circle,draw, fill=black, inner sep=0pt,minimum size=5pt] at ($(father6)+(0,0)$){};
%\node[circle,draw, fill=black, inner sep=0pt,minimum size=5pt] at ($(father7)+(0,0)$){};
%\node[circle,draw, fill=black, inner sep=0pt,minimum size=5pt] at ($(father8)+(0,0)$){};
\node[circle,draw, fill=black, inner sep=0pt,minimum size=5pt] at ($(father9)+(0,0)$){};
%\node[circle,draw, fill=black, inner sep=0pt,minimum size=5pt] at ($(father11)+(0,0)$){};
\node[circle,draw, fill=black, inner sep=0pt,minimum size=5pt] at ($(father13)+(0,0)$){};

%%%%%% level 1

\node[circle,draw, fill=white, inner sep=0pt,minimum size=5pt] at ($(son1)+(0,0)$){};

%\node[circle,draw, fill=black, inner sep=0pt,minimum size=5pt] at ($(son1a)+(0,0)$){};
%\node[circle,draw, fill=black, inner sep=0pt,minimum size=5pt] at ($(son1b)+(0,0)$){};
%\node[circle,draw, fill=black, inner sep=0pt,minimum size=5pt] at ($(son1c)+(0,0)$){};
\node[circle,draw, fill=black, inner sep=0pt,minimum size=5pt] at ($(son2)+(0,0)$){};
\node[circle,draw, fill=white, inner sep=0pt,minimum size=5pt] at ($(son4)+(0,0)$){};
\node[circle,draw, fill=white, inner sep=0pt,minimum size=5pt] at ($(son10)+(0,0)$){};
%\node[circle,draw, fill=black, inner sep=0pt,minimum size=5pt] at ($(son11)+(0,0)$){};
\node[circle,draw, fill=black, inner sep=0pt,minimum size=5pt] at ($(son12)+(0,0)$){};
%\node[circle,draw, fill=black, inner sep=0pt,minimum size=5pt] at ($(son13)+(0,0)$){};
\node[circle,draw, fill=white, inner sep=0pt,minimum size=5pt] at ($(son14)+(0,0)$){};

%%%%%% level 2
\node[circle,draw, fill=white, inner sep=0pt,minimum size=5pt] at ($(grandson3)+(0,0)$){};
%\node[circle,draw, fill=black, inner sep=0pt,minimum size=5pt] at ($(grandson4)+(0,0)$){};
\node[circle,draw, fill=black, inner sep=0pt,minimum size=5pt] at ($(grandson5)+(0,0)$){};
%\node[circle,draw, fill=black, inner sep=0pt,minimum size=5pt] at ($(grandson6)+(0,0)$){};
\node[circle,draw, fill=white, inner sep=0pt,minimum size=5pt] at ($(grandson7)+(0,0)$){};

%%%%%% level 2
%\node[circle,draw, fill=white, inner sep=0pt,minimum size=5pt] at ($(ggson3)+(0,0)$){};
%\node[circle,draw, fill=black, inner sep=0pt,minimum size=5pt] at ($(ggson4)+(0,0)$){};
%\node[circle,draw, fill=white, inner sep=0pt,minimum size=5pt] at ($(ggson5)+(0,0)$){};
%\node[circle,draw, fill=black, inner sep=0pt,minimum size=5pt] at ($(ggson6)+(0,0)$){};
%\node[circle,draw, fill=white, inner sep=0pt,minimum size=5pt] at ($(ggson7)+(0,0)$){};

%%%%%%%%%%% LEGEND

%\draw[blue, very thick] at (origin) rectangle (3,2);
%\node[rectangle,draw] at (origin1) {};

\node[circle,draw, fill=white, inner sep=0pt,minimum size=5pt] at (origin1){};
\node at ($(origin1) +(0.35,0.01)$)  {Dirichlet nodes};

\node[circle,draw, fill=black, inner sep=0pt,minimum size=5pt] at (origin2){};
\node at ($(origin2) +(0.3,0.01)$)  {Active nodes};

\node[rectangle, inner sep=0pt, minimum height=3pt, minimum width=6pt, draw=red, fill=red] at (origin3)  {};
%\node at (origin3)  {\Cross};
\node at ($(origin3) +(0.4,0)$)  {Removable basis};

%%%%%%%%%%%  TEXT

%%%% left  %%%%

%\node[anchor=west] at ($(father1)-(0.9,0)$) {level 0 (base)};
%\node[anchor=west] at ($(son1)-(0.75,0)$) {level 1};
%\node[anchor=west] at ($(grandson1)-(0.75,0)$) {level 2};
%\node[anchor=west] at ($(ggson1)-(0.75,0)$) {level 3};
%\node[anchor=west] at ($(gggson1)-(0.9,0)$) {Overlapped mesh};

%%%% bottom  %%%%

\node at ($(gggson3) +(0,-0.2)$)  {$p=3$};
\node at ($(gggson6) +(0,-0.2)$)  {$p=2$};
\node at ($(gggson8) +(0,-0.2)$)  {$p=3$};
\node at ($(gggson13) +(0,-0.2)$)  {$p=3$};
\node at ($(gggson19) +(0,-0.2)$)  {$p=1$};
\node at ($(gggson23) +(0,-0.2)$)  {$p=1$};



\end{tikzpicture}
\end{center}
}
}
 %
\block{Problem settings}{ %
We consider the abstract variational formulation and its discrete version:
\begin{quote}
Find $u \in \H$ and $u_{\calF} \in \H_{\calF}$ such that
%
\begin{align}
%\SwapAboveDisplaySkip
b(u,\phi)=f(\phi), \quad \forall \phi\in \H&;&b(u_{\calF},\phi_{\calF}) = f(\phi_{\calF}), \quad \forall \phi_{\calF} \in \H_{\calF}.%\label{direct}%\\
%b(u,v^*)=\alert<1>{l(u)}&\quad \forall u\in \H, \label{adjoint}
\end{align}
%where $\H$ is a Hilbert functional space and $\H_{\calF}$ is a finite element discretization $\calT$ of $\H$, such that $\H_{\calF} \subset \H$.
\end{quote}
%
\innerblock{}{
\begin{quote}
The objective is to control
%
\begin{align}
\abs{l(u_\calF)-l(u_{\calE_K})}, \quad \forall K\in \calT,
\end{align}
where $l: \H \longrightarrow \R$ is a linear continuous form.
\end{quote}
}
We introduce the following adjoint problem and its discrete counterpart as follows:
\begin{quote}
Find $v \in \H$ and $v_{\calF}\in \H_{\calF}$ such that
%
\begin{align}
b(\phi,v)=l(\phi), \quad \forall \phi\in \H&;&b(\phi_{\calF},v_{\calF})=l(\phi_{\calF}), \quad \forall \phi_{\calF} \in \H_{\calF}.
\end{align}
%where $v_{\calF}$ stands for the Galerkin approximation of the solution $v$ to the adjoint problem associated with the space $\H_{\calF}$.
\end{quote}
}
%
\block{Error indicators}{ %
\innerblock{}{
We define the element-wise indicators for energy-norm based elliptic problems as
 \begin{equation}
  \eta_K\coloneqq \norm{\Pi_{\calF}^{\calR_K} u_{\calF}}_e^2, \quad \forall K \in \calT.\label{eq:SPDindicators}
\end{equation}
}
\innerblock{}{
We define the element-wise indicators for goal-oriented problems as
\begin{equation}
  \eta_K \coloneqq \abs{a(\Pi_{\calF}^{\calR_K} u_{\calF},\Pi_{\calF}^{\calR_K} v_{\calF})}, \quad \forall K\in \calT.
\end{equation}
}
}
%
 \column{0.5} 
 %
\block{An indefinite problem}{%
Find $u$ such that, 
		\begin{alignat*}{2}
			- \nabla \cdot ( \sigma \nabla u) - k^{2}u &= \scalar{\mathds{1}_{(0,\frac{1}{4})^{2}}}{\cdot}_{L^{2}(\Omega)} \,\, && \text{ in } \Omega \subset \R^{2},\\
			u&=0 \,\, && \text{ on } {\color{blue}{\Gamma_{D}}},\\
			\grad u\cdot \vec{n} &= 0 \,\, && \text{ on } {\color{red}{\Gamma_{N}}}.
		\end{alignat*}%
\begin{align}
\SwapAboveDisplaySkip
	l(\cdot)=\scalar{\mathds{1}_{(\frac{3}{4},1)^{2}}}{\cdot}_{L^{2}(\Omega)};&&a(\cdot,\cdot) =& \, \sum_{K} \abs{\scalar{\grad \cdot}{\grad \cdot}_{L^{2}(K)} + \abs{k^{2}} \scalar{\cdot}{\cdot}_{L^{2}(K)}}.
\end{align}

\begin{multicols}{2}
%\begin{subcolumns}
%\subcolumn{0.333}
\innerblock{$k = (0, 0)$}{
\begin{center}
 \begin{tikzpicture}[scale=2.89,x=2.5cm,y=2.5cm]
    \node(n1) at (0.25,0){};
    \node(n2) at (0.75,0){};
    \node(n3) at (0.75,0.25){};

    \node(n4) at (0.25,0.25){};
    \node(n5) at (0.5,0.5){};
    \draw (n4) rectangle (n5) node[pos=0.5] {$\Omega_{f}$};

    \node(n7) at (0.75,0.75){};
    \draw (n5) rectangle (n7) node[pos=0.5] {$\Omega_{l}$};

    \draw[color=blue, thick,] (0.25,0) -- (0.75,0) -- (0.75,0.25) -- (1,0.25) -- (1,0.75) -- (0.75,0.75) -- (0.75,1) -- (0.25,1) -- (0.25,0.75) -- (0,0.75) -- (0,0.25) -- (0.25,0.25) -- (0.25,-0.005);
    \node(omega) at (0.5,0.90){$\Omega$};
    \draw[color=blue, thick] (n1.center) --  (n2.center) -- node[pos=0.5,pin={0: $\Gamma_{D}$}]{} (n3.center);
  \end{tikzpicture}
\end{center}
}
\innerblock{$k = (4 \cdot 2 \pi, 2 \pi)$}{
\begin{center}
\begin{tikzpicture}[scale=2.75,x=2.5cm,y=2.5cm]
    	\node(n1) at (0,0){};
    	\node(n2) at (1,0){};
    	\node(n3) at (1,1){};
    	\node(n4) at (0,1){};

    	\node(n5) at (0.25,0.25){};
    	\node(n6) at (0.75,0.25){};
    	\node(n7) at (0.75,0.75){};
    	\node(n8) at (0.25,0.75){};

    	\draw (n1) rectangle (n5) node[pos=0.5] {$\Omega_{f}$};
    	\draw (n7) rectangle (n3) node[pos=0.5] {$\Omega_{l}$};

    	\draw[color=blue, thick] (n5) rectangle (n7);

    	\draw[color=blue, thick] (n4.center) -- node[pos=0.5,pin={180: $\Gamma_{D}$}]{} (n1.center) -- (n2.center);
    	\draw[color=red, thick] (n4.center)--  (n3.center) -- node[pos=0.5,pin={0: $\Gamma_{N}$}]{} (n2.center);

    	\node(omega) at (0.5,0.85){$\Omega$};

  	\end{tikzpicture}
\end{center}
}
%\subcolumn{0.333}
%\end{subcolumns}
\end{multicols}
}

\block{Final $hp$-adapted meshes for Poisson example}{%
\plothp{CrossGOA}
}

\block{Final $hp$-adapted meshes for Helmholtz example}{%
\plothp{Helm2DGOA}
}
\end{columns} 
% 
%\note[targetoffsetx=-3cm, targetoffsety=-2cm, angle=0, radius=0cm,width=7cm, rotate=10, connection, linewidth=.2cm,
%     roundedcorners=30, innersep=1cm]{\centering\vspace{1.5cm}\textbf{Published!}
%\scriptsize
%%Felipe V. Caro and Vincent Darrigrand and Julen Alvarez-Aramberri and Elisabete Alberdi and David Pardo.\\
%%A painless multi-level automatic goal-oriented $hp$-adaptive coarsening strategy for elliptic and non-elliptic problems.\\
%{\em Computer Methods in Applied Mechanics and Engineering}, 2022.
% }
\end{document}
%
% \block{Convergence}{%
% \begin{tikzfigure}[Evolution of the error in the QoI.]
%	\errorplotlog{Helm2DGOA}{Error}
%\end{tikzfigure}
%}
%%
%\note[targetoffsetx=-3cm, targetoffsety=-2cm, angle=0, radius=0cm,width=7cm, rotate=10, connection, linewidth=.2cm,
%     roundedcorners=30, innersep=1cm]{\centering\vspace{1.5cm}\textbf{Published!}
%\scriptsize
%%Felipe V. Caro and Vincent Darrigrand and Julen Alvarez-Aramberri and Elisabete Alberdi and David Pardo.\\
%%A painless multi-level automatic goal-oriented $hp$-adaptive coarsening strategy for elliptic and non-elliptic problems.\\
%{\em Computer Methods in Applied Mechanics and Engineering}, 2022.
% }
%
%\begin{columns}
%
%\column{0.25}
%\block{Formations}{
%\begin{itemize}
%\item Teaching (192h eq. TD in 2014-2015);
%\item Various formations on scientific computing (python, fortran, MPI);
%\item Languages certifications (DELE-C1 and CAE).
%\end{itemize}
%
%}
%\column{0.5}
%
%\block{Work in progress and future work}{
%\begin{itemize}
%\item Multi-D Goal-oriented Adaptivity;
%\item Theoretical proof of if there exists a $\hat{b}$ such that the alternative estimate is sharper than the classical one;
%\item Widen the study to other kind of problems (eg. diffusion-convection problems);
%\item Application to non-linear quantities of interest.
%\end{itemize}
%%
%}
%\column{0.25}
%
%
%\block{Mobility}{
%\begin{itemize}
%\item Bilbao (18 months);
%\item Pau (12 months);
%\item Valparaiso (6 months).
%\end{itemize}
%%
%}
%
%\block{}{
%\begin{center}
%\begin{tikzpicture}
%\begin{scope}[it_style/.style={rectangle, fill=mygray!20,line width=4pt,}]
%
%\node(phantom){};
%\shade[ball color=mygreen!30] (phantom) ellipse [x radius=30cm, y radius=8cm];
%\node[left=of phantom](WIP){};
%\node[left=15 of WIP](phantom_wip){};
%\node[above=of phantom_wip](MDGOA){};
%\node[above=of MDGOA](PROOF){};
%\node[below=of phantom_wip](Probs){};
%\node[below=of Probs](NLQOI){};
%
%\draw (MDGOA) node [it_style] {Multi-D Goal-oriented Adaptivity};
%\path (PROOF) node [it_style] {Theoretical proof};
%\path (Probs) node [it_style] {Widen the range of problems};
%\path (NLQOI) node [it_style] {Application to non-linear QoI};
%
%\node[right=15 of phantom](Mob){};
%\node[above=of Mob](BIO){};
%\node[below=of Mob](VALPO){};
%
%\path (Mob) node [it_style] {Pau (12 monthes)};
%\path (BIO) node [it_style] {Bilbao (18 monthes)};
%\path (VALPO) node [it_style] {Valparaiso (6 monthes)};
%
%\end{scope}
% \begin{scope}[mindmap,
%every node/.style={concept, circular drop shadow,execute at begin node=\hskip0pt, scale=1},
%%root concept/.style={concept color=mygreen!50,faded/.style={concept color=mygreen!50}},
%level 1 concept/.append style={level distance=0.1\textwidth},
%level 2 concept/.append style={level distance=0.07\textwidth},
%WIP/.style={concept color=mybrown!50},
% ]
% \node[root concept]{}% root
% child[grow=180,WIP]{node{Work in Progress} 
% [clockwise from=-90]
%  child{node{Multi-D Goal-oriented Adaptivity}}
%     child { node{Theoretical proof}}
%    child{node{Application to non-linear QoI}}
%    child{node[ellipse]{Widen the range of problems}}
%    }
%child[grow=0,WIP]{node{Mobility} 
% [clockwise from=60]
%  child{node{Bilbao (18 monthes) }}
%     child{node{Pau (12 monthes)}}
%    child{node{Valparaiso (6 monthes)}}
%    };
%  \end{scope}
%\end{tikzpicture}
%\end{center}
%
%}

%\begin{columns}
%%%
%  %%
% \column{0.5} 
% %%
% %%
% 
% \block{Work in progress}{
% \begin{itemize}
%\item Application to multi-D goal oriented adaptive procedure
%\item 
%\end{itemize}
%}
%%%%%%%%%%%%%
%%%
%  %%
% \column{0.5} 
% %%
% %%
% %%%%%%%%%%%%
% 
%\block{Formations and Mobility}{blabla}
%%%%%%%%%%%%%
%
 %\end{columns} 
 
% \innerblock{3D}{
%$$k = (4 \cdot 2 \pi, 2 \pi).$$
%${\color{blue}{\Gamma_{D}}} =\text{ faces whose intersection is (0,0,0) }.$
%${\color{red}{\Gamma_{N}}}   =\text{ faces whose intersection is (1,1,1) }.$
%\begin{center}
%\tdplotsetmaincoords{60}{125}
%\begin{tikzpicture}%[x=0.3\colwidth]
%	[tdplot_main_coords,
%		grid/.style={very thin,gray},
%		axis/.style={->,black,thin},
%		cube/.style={ very thick},
%		cube hidden/.style={very thick,dashed},
%		QoI/.style={very thick,fill=green},%,pattern=custom north west lines, hatchcolor=green!100!black,opacity=1},
%		%x=-2cm,
%		%y=2cm,
%		%z=2cm,
%		scale=3.8
%		]
%	%draw a grid in the x-y plane
%%	\foreach \x in {-0.5,0,...,2.5}
%%		\foreach \y in {-0.5,0,...,2.5}
%%		{
%%			\draw[grid] (\x,-0.5) -- (\x,2.5);
%%			\draw[grid] (-0.5,\y) -- (2.5,\y);
%%		}			
%
%	%draw the axes
%	\draw[axis] (1,0,0) -- (1.5,0,0) node[anchor=west]{$x$};
%	\draw[axis] (0,1,0) -- (0,1.5,0) node[anchor=west]{$y$};
%	\draw[axis] (0,0,1) -- (0,0,1.5) node[anchor=west]{$z$};
%
%	%draw the bottom of the cube
%	\draw[cube,dashed, opacity=1] (0,0,0) -- (0,1,0) -- (1,1,0) -- (1,0,0) -- cycle;
%	
%	%draw the back-right of the cube
%	\draw[cube,dashed,opacity=1] (0,0,0) -- (0,1,0) -- (0,1,1) -- (0,0,1) -- cycle;
%
%	%draw the back-left of the cube
%	\draw[cube,dashed,opacity=1] (0,0,0) -- (1,0,0) -- (1,0,1) -- (0,0,1) -- cycle;
%
%	%\node[pin={[pin distance=1cm,pin edge={<-,red,ultra thick,dashed}] 70:$\color{red}{\Gamma_D}$}] at (0,0.5,0.8) {};
%	\node at (0.7,0.6,0.5){$\Omega$};
%	%draw the front-right of the cube
%	\draw[cube,fill=blue ,opacity=0.5] (1,0,0) -- (1,1,0) -- (1,1,1) -- (1,0,1) -- cycle;
%
%	%draw the front-left of the cube
%	\draw[cube,fill=blue,opacity=0.5] (0,1,0) -- (1,1,0) -- (1,1,1) -- (0,1,1) -- cycle;
%
%	%draw the top of the cube
%	\draw[cube,fill=blue,opacity=0.5] (0,0,1) -- (0,1,1) -- (1,1,1) -- (1,0,1) -- cycle;
%	\draw[cube] (0,0,1) -- (0,1,1) -- (1,1,1) -- (1,0,1) -- cycle;
%	
%	%\node[pin={[pin distance=1cm,pin edge={<-,solid,blue,ultra thick}] 110:$\color{blue}{\Gamma_N}$}] at  (1/2,0.05,1) {};
%	
%	%draw the QoI
%	%\draw[QoI] (0.9,0.9,1) -- (0.9,1,1) -- (1,1,1) -- (1,0.9,1) -- cycle;
%%	%draw the QoI
%	%\draw[QoI] (1,0.9,0.9) -- (1,0.9,1) -- (1,1,1) -- (1,1,0.9) -- cycle;
%%	%draw the QoI
%	%\draw[QoI] (0.9,1,0.9) -- (0.9,1,1) -- (1,1,1) -- (1,1,0.9) -- cycle;
%	
%	%\node[pin={[pin distance=1.7cm,pin edge={<-,solid,green,ultra thick}] 0:$\color{green!50!black}{QoI}$}] at  (1,1,1) {};
%	
%	%\node[anchor=south, color=blue] at (0,1/2,1) {$\Gamma_N$};
%	%\node[anchor=south, color=red] at (1/2,0,1) {$\Gamma_D$};
%	%\node[anchor=west, color=green!50!black] at (0,1,1/2) {$QoI$};
%	
%\end{tikzpicture}
%%}
%\end{center}
%%\vspace{.5cm}
