% !TEX root =  unref_general.tex
%#1 is the path for the data
%#2 is the iteration for the mesh around 1% error
%#3 is the iteration for the h mesh around 1% error
%#4 column name of the error
\newcommand{\PbName}{}
\newcommand{\ColName}{}

\newcommand{\FigurePath}{Figures}
\newcommand{\DataPath}{}

%:\errorplotlog
\newcommand{\errorplotlog}[3][]{%Aquí empieza la DEFINICION del comando \errorplotlog, el cual tiene 3 argumentos

  \renewcommand{\DataPath}{\FigurePath/#2}

  \renewcommand{\PbName}{#2} %Problem name
  \renewcommand{\ColName}{#3} %Column name

  %\findmax{\DataPath/outputs.txt}{nr_dof}{\maxdof}
  %\findmin{\DataPath/outputs.txt}{nr_dof}{\mindof}
  %\lastelement{\DataPath/outputs.txt}{nr_dof}{\lastdof}
  %\lastelement{\DataPath/outputs.txt}{error_H1_exact}{\lasterror}
  %\findmaxDbl{\DataPath/outputs.txt}{error_H1_exact}{\maxerror}
  %\findminDbl{\DataPath/outputs.txt}{#2}{\minerror}

  \tikzset{/tikz/external/export next=false}
  \begin{tikzpicture}
    \pgfplotsset{xmode=log}

    \begin{axis}[name=mainerrorplot,
        xlabel={Number of DoFs, $N$ (log scale)}, %Eje x
        %x dir=reverse,
        ylabel=Relative error in \% (log scale), %Eje y
        ymode=log,
        xmode=log,
        %ymin=1e-2,
        %ymax=3e1,
        %xmax=5e4,
        %xmin=5e0,
        %yticklabel pos=right,
        width=\textwidth,height=\plotheight,
        ylabel near ticks,
        xlabel near ticks,
        enlargelimits=true,
        legend style={draw=black,fill=white,legend cell align=left, at={(0.5,1.01)}, anchor=south}, %Celda sobre plot (d)
        legend columns=-1
        %legend columns=4,
        %transpose legend,
        %legend columns=10,
        %reverse legend,
      ]
      %
      % Notas (Felipe): #1 es el parámetro para el método y #2 es el nombre de la carpeta del problema \PbName
      %%%%%%%%%%%%%%% Inicia opciones en legend %%%%%%%%%%%%%%%%%%%%%%%%%%%%%%%%%
      \foreach\method in {#1}{												%#1 es el parámetro para el método
          \ifthenelse{\equal{\method}{h}}{
            \foreach \order in {1,2}{
                \edef\temp{\noexpand\addlegendentry{$h$ ($p=\order$)}}
                \pgfplotstableread{\FigurePath/\PbName/h/order_\order/outputs.txt}\loadedtable
                \addplot+[ line width=1pt] table[x expr=\thisrow{nr_dof},y expr=\thisrow{\ColName}] {\loadedtable};
                \temp
              }
          }{}
          \ifthenelse{\equal{\method}{hp}}{
            \foreach \jdex in {1}{
                \edef\temp{\noexpand\addlegendentry{$hp$ $(p+2)$}}
                \pgfplotstableread{\FigurePath/\PbName/hp/order_1/outputs.txt}\loadedtable
                \addplot+[ line width=1pt] table[x expr=\thisrow{nr_dof},y expr=\thisrow{\ColName}] {\loadedtable};
                \temp
              }
          }{}
          \ifthenelse{\equal{\method}{p}}{
            \edef\temp{\noexpand\addlegendentry{$p$}}
            \pgfplotstableread{\FigurePath/\PbName/p/outputs.txt}\loadedtable
            \addplot+[ line width=1pt] table[x expr=\thisrow{nr_dof},y expr=\thisrow{\ColName}] {\loadedtable};
            \temp
          }{}
        }

    \end{axis}
  \end{tikzpicture}
}%Aquí TERMINA la DEFINICION del comando \errorplotlog

%%%%%%%%%%%%%%%%%%%%%%%%\computeLinearCoef %%%%%%%%%%%%%%%%%%%%%%%%%%%%%%
%\newcommand{\computeLinearCoef}[2]{
%    \lastelement{#1}{iter}{\lastiter}
%    \pgfmathsetmacro\iterInter{\lastiter-3}
%	\getelemdbl{#1}{Error}{\iterInter}{\error}
%	\getelem{#1}{nr_dof}{\iterInter}{\dof}
%
%	\pgfmathsetmacro\iterInter{\lastiter-1}
%	\getelemdbl{#1}{Error}{\iterInter}{\errorTwo}
%	\getelem{#1}{nr_dof}{\iterInter}{\dofTwo}
%
%	\pgfmathsetmacro#2{(ln(\error)-ln(\errorTwo))/(ln(\dof)-ln(\dofTwo))}
%}
%%%%%%%%%%%%%%%%%%%%%%%%\computeLinearCoef %%%%%%%%%%%%%%%%%%%%%%%%%%%%%%

%\newcommand{\plothp}[2][]{
%\renewcommand{\DataPath}{\FigurePath/#2/hp/p2/unref_0.1/unref_h_0.3}

\newcommand{\plotmultisolmesh}[5][]{



  % \foreach \eachiter in {7,10,16,27}{
  \foreach \eachiter in {#3#4#5}{
      \FormatIntegerThreeDigit{\eachiter}{\eachiterformated}
      \FormatIntegerTwoDigit{\eachiter}{\eachiterformatedTwo}
      %
      \begin{subfigure}{\subplotwidth}
        \plothpsolsol{#2/hp/order_1}{\eachiterformatedTwo}{Direct_solution_real}{unitsquare}
        \caption{Solution at iteration \eachiter.}
        \label{fig:#2_\eachiterformated_iter_sol}
      \end{subfigure}
      \begin{subfigure}{\subplotwidth}
        \plothpmeshone{#2/hp/order_1}{\eachiterformated}{X}{\eachiter}
        \caption{$hp$-adapted mesh at iteration \eachiter.}
        \label{fig:#2_\eachiterformated_iter_mesh}
      \end{subfigure}
      % \begin{subfigure}{\subplotwidth}
      %   \plothpmeshone{#2/hp/order_1}{\eachiterformated}{Y}{\eachiter}
      %   \caption{$hp$-adapted mesh iteration: \eachiter.}
      %   \label{fig:#2_\eachiterformated_iter_mesh}
      % \end{subfigure}
    }
  %
}



%%%%%%%%%%%%%%%%%%%%%%%%%%%%%%%%%%%%%%%%%%%%%%%%%%%%%%%%%%%%%%%%%%%%%%%%%%%%%%%%%%%%%%
% DEFINICION de \plothpsinglemesh. Tiene 2 argumentos, si no se especifica un valor en el argumento queda un espacio en blanco por default.
\newcommand{\plothpsinglemesh}[2][]{

  \renewcommand{\DataPath}{\FigurePath/#2/hp/order_1}%\PbName/hp
  \findmax{\DataPath/outputs.txt}{iter}{\endIter}

  \FormatIntegerThreeDigit{\endIter}{\endIterformated}

  \foreach \eachiter in {1,2,3,4,5,6}{
      \FormatIntegerThreeDigit{\eachiter}{\eachiterformated}
      \begin{subfigure}{\subplotwidth}
        \plothpmeshone{#2/hp/order_1}{\eachiterformated}{X}{\eachiter}
        \caption{$hp$-adapted mesh iteration: \eachiter.}
        \label{fig:#2_\eachiterformated_iter}
      \end{subfigure}
    }
}
% DEFINICION de \plothpmeshone. Por definir el numero de argumentos.
\newcommand{\plothpmeshone}[4]{
  \renewcommand{\DataPath}{\FigurePath/#1}

  \lastelement{\DataPath/outputs.txt}{MaxP}{\maxp}
  \lastelement{\DataPath/outputs.txt}{MinP}{\minp}

  %\pgfplotsset{colormap/jet} %Barra de color
  \pgfplotsset{colormap/YlOrRd} %Barra de color
  \tikzset{/tikz/external/export next=false}
  \begin{tikzpicture}
    \begin{axis}[name=master,
        width=\textwidth,height=\textwidth,
        enlargelimits=false,
        xlabel=$x$,
        ylabel=$y$,
        xlabel near ticks,
        ylabel near ticks,
        colorbar horizontal,
        colorbar style={%
            xtick={1,2,...,\maxp},
            title={Order of approximation},
            at={(master.above north west)},anchor=south west,
            yshift=2pt,
            title style={
                yshift=2pt
              },
            yshift=0.2em,
            xticklabel pos=upper,
          },
        %point meta min=\minp,
        point meta min=1,
        point meta max=\maxp,
      ]

      \addplot [forget plot] graphics[xmin=0,xmax=1,ymin=0,ymax=1] {\DataPath/mesh_#3_#2.png};

    \end{axis}
  \end{tikzpicture}
}

%%%%%%%%%%%%%%%%%%%%%%\PLOT_HP_MESH %%%%%%%%%%%%%%%%%%%%%%%%%%%%%%
%%%%%%%%%%%%%%%%%%%%%%: \plothpmesh{#2/hp}{\endIterformated}{\idim} \idim in {X,Y}
\newcommand{\plothpmesh}[3]{
  \renewcommand{\DataPath}{\FigurePath/#1}

  \lastelement{\DataPath/outputs.txt}{MaxP}{\maxp}
  \lastelement{\DataPath/outputs.txt}{MinP}{\minp}

  %\pgfplotsset{colormap/jet} %Barra de color
  \pgfplotsset{colormap/YlOrRd} %Barra de color
  \tikzset{/tikz/external/export next=false}
  \begin{tikzpicture}
    \begin{axis}[name=master,
        width=\textwidth,height=\textwidth,
        enlargelimits=false,
        xlabel=$x$,
        ylabel=$y$,
        xlabel near ticks,
        ylabel near ticks,
        colorbar horizontal,
        colorbar style={%
            xtick={1,2,...,\maxp},
            title={Order of approximation},
            at={(master.above north west)},anchor=south west,
            yshift=2pt,
            title style={
                yshift=2pt
              },
            yshift=0.2em,
            xticklabel pos=upper,
          },
        %point meta min=\minp,
        point meta min=1,
        point meta max=\maxp,
      ]
      %
      \addplot [forget plot] graphics[xmin=0,xmax=1,ymin=0,ymax=1] {\DataPath/mesh_#3_#2.png}; %Li­mites del dominio de los subplots (a) y (b). #3 X o Y, #2 por ejemplo 015

    \end{axis}
  \end{tikzpicture}
}
%%%%%%%%%%%%%%%%%%%%%%%%%% \PLOT_HP_MESH%%%%%%%%%%%%%%%%%%%%%%%%%%%%%%%

%%%%%%%%%%%%%%%%%%%%%%\PLOT_HP_LEVEL %%%%%%%%%%%%%%%%%%%%%%%%%%%%%%
%%%%%%%%%%%%%%%%%%%%%%: \plothplevel{#2/hp}{\endIterformated}{\idim}{ \idim in {X,Y,Z}
\newcommand{\plothplevel}[3]{
  \renewcommand{\DataPath}{\FigurePath/#1}

  \lastelement{\DataPath/outputs.txt}{MaxP}{\maxp}
  \lastelement{\DataPath/outputs.txt}{MinP}{\minp}

  %\pgfplotsset{colormap/jet} %Barra de color
  \pgfplotsset{colormap/YlOrRd} %Barra de color
  \tikzset{/tikz/external/export next=false}
  \begin{tikzpicture}
    \begin{axis}[name=master,
        width=\textwidth,height=\textwidth,
        enlargelimits=false,
        %xlabel=$x$,
        %ylabel=$y$,
        %xlabel near ticks,
        %ylabel near ticks,
        %axis line style={draw=none},
        %tick style={draw=none},
        hide axis,
        xticklabels={},
        yticklabels={},
        colorbar horizontal,
        colorbar style={%
            xtick={1,2,...,\maxp},
            title={Order of approximation},
            at={(master.above north west)},anchor=south west,
            yshift=10pt,
            title style={
                yshift=2pt
              },
            yshift=0.2em,
            xticklabel pos=upper,
          },
        %point meta min=\minp,
        point meta min=1,
        point meta max=\maxp,
      ]
      %
      \addplot [forget plot] graphics[xmin=0,xmax=1,ymin=0,ymax=1] {\DataPath/boxes/box_mesh_#3_#2.png};

    \end{axis}
  \end{tikzpicture}
}
%%%%%%%%%%%%%%%%%%%%%%%%%% \PLOT_HP_LEVEL%%%%%%%%%%%%%%%%%%%%%%%%%%%%%%%

%%%%%%%%%%%%%%%% Aquí EMPIEZA la DEFINICION del COMANDO \PLOT_H_MESH el cual tiene 2 argumentos %%%%%%%%%%
\newcommand{\plothmesh}[2]{%\plothmesh{#2/h/unref_0.1/unref_h_0.3/order_1/}{\endIterhformated}{1}
  \renewcommand{\DataPath}{\FigurePath/#1} %Aquí #1 es el 1er parámetro en la llamada de \plothpmesh, es decir, #2/h/order_1
  % \newcommand{\N}{\mathbb N}

  %\lastelement{\DataPath/outputs.txt}{MaxP}{\maxp}
  %\lastelement{\DataPath/outputs.txt}{MinP}{\minp}

  %\pgfplotsset{colormap/YlOrRd}
  \begin{tikzpicture}
    \begin{axis}[name=master,
        width=\textwidth,height=\textwidth,
        enlargelimits=false,
        xlabel=$x$,
        ylabel=$y$,
        xlabel near ticks,
        ylabel near ticks,
      ]
      %
      \addplot [forget plot] graphics[xmin=0,xmax=1,ymin=0,ymax=1] {\DataPath/grid_#2.png}; %Límites del dominio de los subplots (c). %#2 es \endIterformated, un número de 2/3 dígitos, depende de la rutina
      %
      %\IfSubStr{#1}{square}{
      %}{
      %\addplot [forget plot] graphics[xmin=0,xmax=1,ymin=0,ymax=1] {\DataPath/coarse_mesh_#2_1.png};
      %}

    \end{axis}
  \end{tikzpicture}
}
%%%%%%%%%%%%%%%%%%%%%%%%%% Aqui­ TERMINA la DEFINICION del COMANDO \PLOTHMESH %%%%%%%%%%%%%%%%%%%%%%%%%%%

%% Aqui empieza la DEFINICION del COMANDO \plothp el cual tiene 2 argumentos, si no se especifica un valor en el argumento queda un espacio en blanco por default.
\newcommand{\plothp}[2][]{

  \renewcommand{\DataPath}{\FigurePath/#2/hp/order_1}%\PbName/hp
  \findmax{\DataPath/outputs.txt}{iter}{\endIter}
  %\findmin{\DataPath/outputs.txt}{iter}{\initIter}

  %\pgfmathparse{subtract(#2,0)}
  %\pgfmathtruncatemacro\index{\pgfmathresult}
  %\getelem{\DataPath/outputs.txt}{nr_dof}{\index}{\nrdof}
  %\getelemdbl{\DataPath/outputs.txt}{Error_H1}{\index}{\error}

  \FormatIntegerThreeDigit{\endIter}{\endIterformated}

  \foreach \idim in {X,Y}{
      \begin{subfigure}{\subplotwidth} %subplot (a) (b)
        \plothpmesh{#2/hp/order_1}{\endIterformated}{\idim}
        \caption{Final $hp$-adapted mesh with polynomial orders in the \ifthenelse{\equal{\idim}{X}}{$x$}{$y$}-direction.}
        \label{fig:#2_\idim_final}
      \end{subfigure}
    }

  %\pgfmathparse{subtract(\endIter,0)}
  %\pgfmathtruncatemacro\endIterIndex{\pgfmathresult}

  %\getelem{\DataPath/outputs.txt}{nr_dof}{\endIter}{\nrdofHpEnd}
  %\getelemdbl{\DataPath/outputs.txt}{Error_H1}{\endIter}{\errorHpEnd}

  %
  \findmax{\FigurePath/#2/h/order_1/outputs.txt}{iter}{\endIterh}


  %\getelemdbl{\FigurePath/#2/h/order_1/outputs.txt}{Error}{\endIterh}{\errorh}
  %\getelem{\FigurePath/#2/h/order_1/outputs.txt}{nr_dof}{\endIterh}{\nrdofh}

  \FormatIntegerThreeDigit{\endIterh}{\endIterhformated}

  \begin{subfigure}{\subplotwidth} %subplot (c)
    \plothmesh{#2/h/order_1}{\endIterhformated}{}
    \caption{Final $h$-adapted mesh, $p=1$.}
    \label{fig:#2_h_final}
  \end{subfigure}
  %
  %\begin{subfigure}{\subplotwidth} %subplot (c)
  %%\plothmesh{#1/h/unref_0.1}{\endIterh}
  %%\FormatIntegerThreeDigit{\endIterh}{\endIterhformated}
  %\FormatIntegerThreeDigit{\endIterh}{\endIterhformated}
  %\plothmesh{#2/h/order_1}{\endIterhformated}{}
  %\caption{Final adapted $h$-mesh, $p=1$.}
  %\label{fig:#2h_final}
  %\end{subfigure}
  %%
  %
  \begin{subfigure}{\subplotwidth} %subplot (d)
    %
    \errorplotlog[#1]{#2}{Error}
    %%
    \subcaption{Evolution of \captionifthenproblem{#2}.}
    \label{fig:#2_error}

  \end{subfigure}
}

%% Aqui­ empieza la DEFINICION del COMANDO }ºNON. Un llamado es por ejemplo \plothpnon[hp,h]{Helm2DGOANon}.
\newcommand{\plothpnon}[2][]{

  \ifthenelse{\equal{#2}{ConvDiff2DGOANon}}{
  	\renewcommand{\DataPath}{\FigurePath/#2/hp/order_1}%\PbName/hp
  \findmax{\DataPath/outputs.txt}{iter}{\endIter}
  %\findmin{\DataPath/outputs.txt}{iter}{\initIter}

  %\pgfmathparse{subtract(#2,0)}
  %\pgfmathtruncatemacro\index{\pgfmathresult}
  %\getelem{\DataPath/outputs.txt}{nr_dof}{\index}{\nrdof}
  %\getelemdbl{\DataPath/outputs.txt}{Error_H1}{\index}{\error}

  \FormatIntegerThreeDigit{\endIter}{\endIterformated}

  \foreach \idim in {X,Y}{
      \begin{subfigure}{\subplotwidth} %subplot (a) (b)
        \plothpmesh{#2/hp/order_1}{\endIterformated}{\idim}
        \caption{Final $hp$-adapted mesh with polynomial orders in the \ifthenelse{\equal{\idim}{X}}{$x$}{$y$}-direction.}
        \label{fig:#2_\idim_final}
      \end{subfigure}
    }

  %\pgfmathparse{subtract(\endIter,0)}
  %\pgfmathtruncatemacro\endIterIndex{\pgfmathresult}

  %\getelem{\DataPath/outputs.txt}{nr_dof}{\endIter}{\nrdofHpEnd}
  %\getelemdbl{\DataPath/outputs.txt}{Error_H1}{\endIter}{\errorHpEnd}

  %
  \findmax{\FigurePath/#2/h/order_1/outputs.txt}{iter}{\endIterh}


  %\getelemdbl{\FigurePath/#2/h/order_1/outputs.txt}{Error}{\endIterh}{\errorh}
  %\getelem{\FigurePath/#2/h/order_1/outputs.txt}{nr_dof}{\endIterh}{\nrdofh}

  \FormatIntegerThreeDigit{\endIterh}{\endIterhformated}

  \begin{subfigure}{\subplotwidth} %subplot (c)
    \plothmesh{#2/h/order_1}{\endIterhformated}{}
    \caption{Final $h$-adapted mesh, $p=1$.}
    \label{fig:#2_h_final}
  \end{subfigure}
  \begin{subfigure}{\subplotwidth} %subplot (d)
    %
    \errorplotlog[#1]{#2}{Lower}
    %%
    \subcaption{Evolution of \captionifthenproblem{#2}.}
    \label{fig:#2_lower}

  \end{subfigure}
  }
  {
  \renewcommand{\DataPath}{\FigurePath/#2/hp/order_1}%\PbName/hp
  \findmax{\DataPath/outputs.txt}{iter}{\endIter}
  %\findmin{\DataPath/outputs.txt}{iter}{\initIter}

  %\pgfmathparse{subtract(#2,0)}
  %\pgfmathtruncatemacro\index{\pgfmathresult}
  %\getelem{\DataPath/outputs.txt}{nr_dof}{\index}{\nrdof}
  %\getelemdbl{\DataPath/outputs.txt}{Error_H1}{\index}{\error}

  \FormatIntegerThreeDigit{\endIter}{\endIterformated}

  \foreach \idim in {X,Y}{
      \begin{subfigure}{\subplotwidth} %subplot (a) (b)
        \plothpmesh{#2/hp/order_1}{\endIterformated}{\idim}
        \caption{Final $hp$-adapted mesh with polynomial orders in the \ifthenelse{\equal{\idim}{X}}{$x$}{$y$}-direction.}
        \label{fig:#2_\idim_final}
      \end{subfigure}
    }

  \findmax{\FigurePath/#2/h/order_1/outputs.txt}{iter}{\endIterh}

  \FormatIntegerThreeDigit{\endIterh}{\endIterhformated}

  \begin{subfigure}{\subplotwidth} %subplot (c)
    \plothmesh{#2/h/order_1}{\endIterhformated}{}
    \caption{Final $h$-adapted mesh, $p=1$.}
    \label{fig:#2_h_final}
  \end{subfigure}
  }

  }


%% Aqui­ empieza la DEFINICION del COMANDO \PLOTHPonemesh.
\newcommand{\plothponemesh}[2][]{

  \renewcommand{\DataPath}{\FigurePath/#2/hp/order_1}%\PbName/hp
  \findmax{\DataPath/outputs.txt}{iter}{\endIter}
  %\findmin{\DataPath/outputs.txt}{iter}{\initIter}

  %\pgfmathparse{subtract(#2,0)}
  %\pgfmathtruncatemacro\index{\pgfmathresult}
  %\getelem{\DataPath/outputs.txt}{nr_dof}{\index}{\nrdof}
  %\getelemdbl{\DataPath/outputs.txt}{Error_H1}{\index}{\error}

  \FormatIntegerThreeDigit{\endIter}{\endIterformated}

  \foreach \idim in {X,Y}{
      \begin{subfigure}{\subplotwidth} %subplot (a) (b)
        \plothpmesh{#2/hp/order_1}{\endIterformated}{\idim}
        \caption{Final $hp$-adapted mesh with polynomial orders in the \ifthenelse{\equal{\idim}{X}}{$x$}{$y$}-direction.}
        \label{fig:#2_\idim_final}
      \end{subfigure}
    }

  %\pgfmathparse{subtract(\endIter,0)}
  %\pgfmathtruncatemacro\endIterIndex{\pgfmathresult}

  %\getelem{\DataPath/outputs.txt}{nr_dof}{\endIter}{\nrdofHpEnd}
  %\getelemdbl{\DataPath/outputs.txt}{Error_H1}{\endIter}{\errorHpEnd}

  %
  \findmax{\FigurePath/#2/h/order_1/outputs.txt}{iter}{\endIterh}


  %\getelemdbl{\FigurePath/#2/h/order_1/outputs.txt}{Error}{\endIterh}{\errorh}
  %\getelem{\FigurePath/#2/h/order_1/outputs.txt}{nr_dof}{\endIterh}{\nrdofh}

  \FormatIntegerThreeDigit{\endIterh}{\endIterhformated}

  \begin{subfigure}{\subplotwidth} %subplot (c)
    \plothmesh{#2/h/order_1}{\endIterhformated}{}
    \caption{Final $h$-adapted mesh, $p=1$.}
    \label{fig:#2_h_final}
  \end{subfigure}
}

%% DEFINICION de \plothpcom
\newcommand{\plothpcom}[2][]{

  \renewcommand{\DataPath}{\FigurePath/#2/hp/order_1}%\PbName/hp
  \findmax{\DataPath/outputs.txt}{iter}{\endIter}

  \FormatIntegerThreeDigit{\endIter}{\endIterformated}

  \findmax{\FigurePath/#2/h/order_1/outputs.txt}{iter}{\endIterh}

  \FormatIntegerThreeDigit{\endIterh}{\endIterhformated}
  %
  \begin{subfigure}{\subplotwidth} %subplot (a)
    %
    \errorplotlog[#1]{#2}{Lower}
    %%
    \subcaption{Evolution of \captionifthenproblem{#2}.}
    \label{fig:#2_lower_com}

  \end{subfigure}
  %
  \hfill
  %
  \begin{subfigure}{\subplotwidth} %subplot (b)
    %
    \errorplotlog[#1]{#2}{Error}
    %%
    \subcaption{Evolution of \captionifthenlegend{#2}.}
    \label{fig:#2_error_com}

  \end{subfigure}
}

% DEFINICION de \plothpGOA (#1 optional parameter which defines the domain)
\newcommand{\plothpGOA}[3][]{

  \renewcommand{\DataPath}{\FigurePath/#2/hp/order_1}
  \findmax{\DataPath/outputs.txt}{iter}{\endIter}

  \FormatIntegerTwoDigit{\endIter}{\endIterformated}
  \FormatIntegerThreeDigit{\endIter}{\endIterformatedthree}
  %
  \begin{subfigure}{\subplotwidth}
    \plothpsolGOA{#2/hp/order_1}{\endIterformated}{Direct_solution_#3}{#1}%
    \subcaption{Solution to the direct problem.}
    \label{fig:#2_dir}
  \end{subfigure}
  \begin{subfigure}{\subplotwidth}
    \plothpsolGOA{#2/hp/order_1}{\endIterformated}{Adjoint_solution_#3}{#1}%
    \subcaption{Solution to the adjoint problem.}
    \label{fig:#2_adj}
  \end{subfigure}
}%
%%%%%%%%%%%%%%%%%%%%%%%%%% \plot %%%%%%%%%%%%%%%%%%%%%%%%%%%%%%%%%.   \captionifthenproblem{#2}

%Aqui­ empieza la DEFINICION del COMANDO \plothpDsol
% #1 optional parameter which defines the domain
\newcommand{\plothpDsol}[2][]{

  \renewcommand{\DataPath}{\FigurePath/#2/hp/order_1}
  \findmax{\DataPath/outputs.txt}{iter}{\endIter}

  \FormatIntegerTwoDigit{\endIter}{\endIterformated}
  \FormatIntegerThreeDigit{\endIter}{\endIterformatedthree}
  %
  \begin{subfigure}{\subplotwidth}
    \plothpsolsol{#2/hp/order_1}{\endIterformated}{Direct_solution_real}{#1}%
  \end{subfigure}
}%

%Aqui­ empieza la DEFINICION del COMANDO \plothpAsol
% #1 optional parameter which defines the domain
\newcommand{\plothpAsol}[2][]{

  \renewcommand{\DataPath}{\FigurePath/#2/hp/order_1}
  \findmax{\DataPath/outputs.txt}{iter}{\endIter}

  \FormatIntegerTwoDigit{\endIter}{\endIterformated}
  \FormatIntegerThreeDigit{\endIter}{\endIterformatedthree}
  %
  \plothpsolsol{#2/hp/order_1}{\endIterformated}{Adjoint_solution_real}{#1}%
}%

%Aqui­ empieza la DEFINICION del COMANDO \plothpSource
% #1 optional parameter which defines the domain
\newcommand{\plothpSource}[2][]{

  \renewcommand{\DataPath}{\FigurePath/#2/hp/order_1}
  \findmax{\DataPath/outputs.txt}{iter}{\endIter}

  \FormatIntegerTwoDigit{\endIter}{\endIterformated}
  \FormatIntegerThreeDigit{\endIter}{\endIterformatedthree}
  %
  \plothpsourcesource{#2/hp/order_1}{\endIterformated}{Source_real}{#1}%
}%


%Aqui empieza la DEFINICION del COMANDO \plothpDmat
\newcommand{\plothpDmat}[2][]{

  \renewcommand{\DataPath}{\FigurePath/#2/hp/order_1}
  \findmax{\DataPath/outputs.txt}{iter}{\endIter}

  \FormatIntegerTwoDigit{\endIter}{\endIterformated}
  \FormatIntegerThreeDigit{\endIter}{\endIterformatedthree}
  %
  \plothpsolsol{#2/hp/order_1}{\endIterformated}{materials}{#1}%
}%
%%%%%%%%%%%%%%%%%%%%%%%%%% \plot %%%%%%%%%%%%%%%%%%%%%%%%%%%%%%%%%.   \captionifthenproblem{#2}

%%%%%%%%%%%%%%%% Aquí EMPIEZA la DEFINICION del COMANDO \plothpsolGOA el cual tiene 2 argumentos %%%%%%%%%%
\newcommand{\plothpsolGOA}[4]{%
  \renewcommand{\DataPath}{\FigurePath/#1}
  \pgfplotsset{%
    colormap={paraview}{rgb=(0.231373, 0.298039, 0.752941) rgb=(0.865003, 0.865003, 0.865003) rgb=(0.705882, 0.0156863, 0.14902)}%
  }%
  \pgfplotsset{colormap name={paraview}}%

  \getelemdbl{\DataPath/#3_#2_ColorbarValues.dat}{min_value}{0}{\scaleMin}
  \getelemdbl{\DataPath/#3_#2_ColorbarValues.dat}{max_value}{0}{\scaleMax}
  \getelemdbl{\DataPath/#3_#2_ColorbarValues.dat}{mean_value}{0}{\scaleMean}

  \tikzset{/tikz/external/export next=false}

  \IfSubStr{#4}{unitsquare}
  {   	\begin{tikzpicture}
      \begin{axis}[name=master,
          width=\textwidth,height=\textwidth,
          enlargelimits=false,
          xlabel=$x$,
          ylabel=$y$,
          xlabel near ticks,
          ylabel near ticks,
          colorbar horizontal,
          colorbar style={%
              xtick={\scaleMin,\scaleMean,\scaleMax},
              title={Value of the solution},
              at={(master.above north west)},anchor=south west,
              yshift=2pt,
              title style={
                  yshift=5pt%yshift=2pt
                },
              yshift=0.2em,
              xticklabel pos=upper,
            },
          %point meta min=\minp,
          point meta min=\scaleMin,
          point meta max=\scaleMax,
        ]
        %
        \addplot [forget plot] graphics[xmin=0,xmax=1,ymin=0,ymax=1] {\DataPath/coarse_mesh_fine_#2_#3.png};

      \end{axis}
    \end{tikzpicture}}
  {\begin{tikzpicture}
      \begin{axis}[name=master,
          width=\textwidth,height=\textwidth,
          enlargelimits=false,
          %xlabel=$x$,
          %ylabel=$y$,
          %xlabel near ticks,
          %ylabel near ticks,
          %axis line style={draw=none},
          %tick style={draw=none},
          hide axis,
          xticklabels={},
          yticklabels={},
          colorbar horizontal,
          colorbar style={%
              xtick={\scaleMin,\scaleMean,\scaleMax},
              title={Value of the solution},
              at={(master.above north west)},anchor=south west,
              yshift=10pt,
              title style={
                  yshift=5pt%yshift=2pt
                },
              yshift=0.2em,
              xticklabel pos=upper,
            },
          %point meta min=\minp,
          point meta min=\scaleMin,
          point meta max=\scaleMax,
        ]
        %
        \addplot [forget plot] graphics[xmin=0,xmax=1,ymin=0,ymax=1] {\DataPath/coarse_mesh_fine_#2_#3.png};

      \end{axis}
    \end{tikzpicture}}%
}
%%%%%%%%%%%%%%%%%%%%%%%%%% Aqui� TERMINA la DEFINICION de \plothpsolGOA %%%%%%%%%%%%%%%

% DEFINICION de \plothpsolsol
\newcommand{\plothpsolsol}[4]{%
  \renewcommand{\DataPath}{\FigurePath/#1}

  \IfSubStr{#4}{unitsquare}
  {   	\pgfplotsset{%
      colormap={paraview}{rgb=(0.231373, 0.298039, 0.752941) rgb=(0.865003, 0.865003, 0.865003) rgb=(0.705882, 0.0156863, 0.14902)}%
    }%
    \pgfplotsset{colormap name={paraview}}%

    \getelemdbl{\DataPath/#3_#2_ColorbarValues.dat}{min_value}{0}{\scaleMin}
    \getelemdbl{\DataPath/#3_#2_ColorbarValues.dat}{max_value}{0}{\scaleMax}
    \getelemdbl{\DataPath/#3_#2_ColorbarValues.dat}{mean_value}{0}{\scaleMean}

    \tikzset{/tikz/external/export next=false}
    \begin{tikzpicture}
      \begin{axis}[name=master,
          width=\textwidth,height=\textwidth,
          % width=\textwidth*0.5,height=\textwidth*0.5,
          enlargelimits=false,
          xlabel=$x$,
          ylabel=$y$,
          xlabel near ticks,
          ylabel near ticks,
          colorbar horizontal,
          colorbar style={%
              xtick={\scaleMin,\scaleMean,\scaleMax},
              title={Value of the solution},
              at={(master.above north west)},anchor=south west,
              yshift=2pt,
              title style={
                  yshift=5pt%yshift=2pt
                },
              yshift=0.2em,
              xticklabel pos=upper,
            },
          %point meta min=\minp,
          point meta min=\scaleMin,
          point meta max=\scaleMax,
        ]
        %
        \addplot [forget plot] graphics[xmin=0,xmax=1,ymin=0,ymax=1] {\DataPath/coarse_mesh_fine_#2_#3.png};
      \end{axis}
    \end{tikzpicture}}
  {%else
    \pgfplotsset{%
      colormap={paraview}{rgb=(0.05, 1.0, 1.0) rgb=(1.0, 0.0, 0.0) rgb=(0.0, 1.0, 0.0) rgb=(0.0, 0.0, 1.0)}%
    }
    %
    \pgfplotsset{colormap name={paraview}}%
    \findmax{\DataPath/#3.txt}{uno}{\matuno}
    \findmax{\DataPath/#3.txt}{dos}{\matdos}
    \findmax{\DataPath/#3.txt}{tres}{\mattres}
    \findmax{\DataPath/#3.txt}{cuatro}{\matcuatro}

    \tikzset{/tikz/external/export next=false}
    \begin{tikzpicture}
      \begin{axis}[name=master,
          width=\textwidth*0.5,height=\textwidth*0.5,
          enlargelimits=false,
          %xlabel=$x$,
          %ylabel=$y$,
          %xlabel near ticks,
          %ylabel near ticks,
          axis line style={draw=none},
          tick style={draw=none},
          xticklabels={},
          yticklabels={},
          colorbar horizontal,
          colorbar sampled,
          colormap access=piecewise constant,
          colorbar style={
              samples=4,
              xtick=data,%={0,1,2,3},
              xticklabels={0.01,1,10,1000},
              xticklabel style={xshift=\textwidth*0.5/10},
              title={Diffusivity of the materials},
              at={(master.above north west)},anchor=south west,
              yshift=5pt,
              title style={
                  yshift=5pt%yshift=2pt
                },
              yshift=0.2em,
              xticklabel pos=upper,
            },
          %point meta min=\minp,
          % point meta min=\matuno,
          % point meta max=\matcuatro,
        ]
        %
        \addplot [forget plot] graphics[xmin=0,xmax=1,ymin=0,ymax=1] {\DataPath/#3_#2.png};

      \end{axis}
    \end{tikzpicture}}%
}

% TERMINA la DEFINICION de \plothpsolsol

% DEFINICION de \plothpsourcesource
\newcommand{\plothpsourcesource}[4]{%
  \renewcommand{\DataPath}{\FigurePath/#1}

  \pgfplotsset{colormap={paraview}{rgb=(0.231373, 0.298039, 0.752941) rgb=(0.865003, 0.865003, 0.865003) rgb=(0.705882, 0.0156863, 0.14902)}}
  \pgfplotsset{colormap name={paraview}}

  \getelemdbl{\DataPath/#3_ColorbarValues.dat}{min_value}{0}{\scaleMin}
  \getelemdbl{\DataPath/#3_ColorbarValues.dat}{max_value}{0}{\scaleMax}
  \getelemdbl{\DataPath/#3_ColorbarValues.dat}{mean_value}{0}{\scaleMean}

  \tikzset{/tikz/external/export next=false}
  \begin{tikzpicture}
    \begin{axis}[name=master,
        width=\textwidth*0.6,height=\textwidth*0.6,
        enlargelimits=false,
        xlabel=$x$,
        ylabel=$y$,
        xlabel near ticks,
        ylabel near ticks,
        colorbar horizontal,
        colorbar style={%
            xtick={\scaleMin,\scaleMean,\scaleMax},
            title={Value of the source},
            at={(master.above north west)},anchor=south west,
            yshift=2pt,
            title style={
                yshift=5pt%yshift=2pt
              },
            yshift=0.2em,
            xticklabel pos=upper,
          },
        %point meta min=\minp,
        point meta min=\scaleMin,
        point meta max=\scaleMax,
      ]
      %
      \addplot [forget plot] graphics[xmin=0,xmax=1,ymin=0,ymax=1] {\DataPath/coarse_mesh_fine_#2_#3.png};
    \end{axis}
  \end{tikzpicture}
}

% TERMINA la DEFINICION de \plothpsource

%%%%%%%%%%%%%%%% Aquí EMPIEZA la DEFINICION del COMANDO \PLOTHPOneMESH el cual tiene 2 argumentos %%%%%%%%%%
\newcommand{\plothpOnemesh}[3]{

  \renewcommand{\DataPath}{\FigurePath/#1}
  \pgfplotsset{colormap/YlOrRd}

  %\getelem{\DataPath/outputs.txt}{iter}{#2}{\iter}
  \findmax{\DataPath/coarse_mesh_#2.txt}{order}{\maxp}
  % \findmin{\DataPath/coarse_mesh_00.txt}{order}{\minp}

  \begin{tikzpicture}
    \begin{axis}[name=master,
        width=\textwidth,height=\plotheight,
        enlargelimits=false,
        enlarge y limits=true,
        enlarge y limits=0.2,
        %point meta min=\minp,
        % point meta min=0,
        % point meta max=\maxp,
        xlabel=$x$,
        ylabel=$u(x)$ ,
      ]

      \addplot[line width=1pt,opacity=0.2] table[x=X,
          y=My_beautiful_solution_real] {\DataPath/sol#2.txt};

      \addplot[solid, line width=1pt, color=black] table[x=X,
          y=My_beautiful_exact_solution_real] {\DataPath/exact_sol.txt};
    \end{axis}

    \begin{axis}[
        colormap access=const,
        colorbar horizontal,
        colorbar sampled,
        colorbar style={ samples=\maxp+1,
            xtick={0.5,1.5,...,\maxp},
            xticklabels={1,2,...,\maxp},
            at={(master.above north west)},anchor=south west,
            title=Approximation order $p$,
            xticklabel pos=upper,
            yshift=1em,
            xtick style={draw=none},
            title style={yshift=2pt},
          },
        width=\textwidth,%
        %at={(master.above north west)},anchor=south west,
        at={(master.south west)},%anchor=south west,
        enlargelimits=false,
        %enlarge x limits=true,
        axis lines=none,
        height=2cm,%\pgfkeysvalueof{/pgfplots/colorbar/width},
        ticks=none,
        %point meta min=\minp,
        point meta min=0,
        point meta max=\maxp,
        title=Mesh,
        name=coarse_mesh,
        title style={left,at={(0,0)}, align=right},
      ]
      \addplot [patch, patch type=rectangle, point meta=explicit,line
        width=0.6pt, faceted color= black]  table[meta expr=\thisrow{order}-0.5]
        {\DataPath/coarse_mesh_#2.txt};%\mesh};

    \end{axis}

  \end{tikzpicture}

}
%%%%%%%%%%%%%%%%%%%%%%%%%% Aquí TERMINA la DEFINICION del COMANDO  %%%%%%%%%%%%%%%

%%%%%%%%%%%%%%%%%%%%%%%%%%%  Aquí empieza la DEFINICION del COMANDO \plothpone
% \plothpone tiene 2 argumentos, si no se especifica un argumento queda un espacio en blanco por default.
% Llamado a la función: \plothpOne{Regular1D}
\newcommand{\plothpOne}[2][]{

  \renewcommand{\DataPath}{\FigurePath/#2/hp/order_1}
  \findmax{\DataPath/outputs.txt}{iter}{\endIter}

  \FormatIntegerTwoDigit{\endIter}{\endIterformated}
  %
  \begin{subfigure}{\subplotwidth}
    \plothpOnemesh{#2/hp/order_1}{\endIterformated}{}%
    \subcaption{\captionifthenproblem{#2} problem.}
  \end{subfigure}
  \begin{subfigure}{\subplotwidth}
    \errorplotlog[#1]{#2}{Error}
    \subcaption{Evolution of the relative error in \captionifthenlegend{#2}.}
  \end{subfigure}
}%
%%%%%%%%%%%%%%%%%%%%%%%%%% \plot %%%%%%%%%%%%%%%%%%%%%%%%%%%%%%%%%

%Aquí empieza la DEFINICION del COMANDO \plothpDupper
% #1 optional parameter which defines the domain
\newcommand{\plothpDupper}[2][]{

  \renewcommand{\DataPath}{\FigurePath/#2/hp/order_1/#1}
  \findmax{\DataPath/outputs.txt}{iter}{\endIter}

  \FormatIntegerTwoDigit{\endIter}{\endIterformated}
  \FormatIntegerThreeDigit{\endIter}{\endIterformatedthree}
  %
  %\begin{figure}
  \plothpupper{#2/hp/order_1/#1}%
  % \subcaption{Upper bounds.}
  % \label{fig:#2_upper}
  %\end{figure}


}%
%%%%%%%%%%%%%%%%%%%%%%%%%% \plot %%%%%%%%%%%%%%%%%%%%%%%%%%%%%%%%%

%%%%%%%%%%%%%%%% Aquí EMPIEZA la DEFINICION del COMANDO \plothpupper %%%%%%%%%%%%%%%
\newcommand{\plothpupper}[1]{
  \renewcommand{\DataPath}{\FigurePath/#1}

  \begin{tikzpicture}
    \begin{axis}[name=master,
        width=\textwidth*0.5,height=\textwidth*0.5,
        enlargelimits=false,
        xlabel={Number of DoFs, $N$ (log scale)},
        %x dir=reverse,
        ylabel=Error,
        ymode=log,
        xmode=log,
        %ymin=1e-2,
        %ymax=3e1,
        %xmax=5e4,
        %xmin=5e0,
        %yticklabel pos=right,
        ylabel near ticks,
        xlabel near ticks,
        enlargelimits=false,
        legend style={draw=black,fill=white,legend cell align=left, at={(0.5,1.01)}, anchor=south},
        legend columns=-1
        %legend columns=4,
        %transpose legend,
        %legend columns=10,
        %reverse legend,
      ]

      \addplot+[] table[x=dofs,y=l_qoi] {\DataPath/bounds.txt} ;
      \addlegendentry{l QoI}

      \addplot+[] table[x=dofs,y=upper_bound_classic_dabs] {\DataPath/bounds.txt} ;
      \addlegendentry{double abs}

      \addplot+[] table[x=dofs,y=upper_bound_classic] {\DataPath/bounds.txt} ;
      \addlegendentry{SPD $(+k^2)$}

      \addplot+[] table[x=dofs,y=alternative_upper_bound] {\DataPath/bounds.txt} ;
      \addlegendentry{$\tilde{\varepsilon}$}

    \end{axis}
  \end{tikzpicture}
}
%%%%%%%%%%%%%%%%%%%%%%%%%% Aquí TERMINA la DEFINICION del COMANDO \plothpupper %%%%%%%%%%%%%%%

%% Aquí empieza la DEFINICION del COMANDO \plothplevels. Un llamado es por ejemplo \plothplevels[hp]{Helm3DGOA}.
\newcommand{\plothplevels}[2][]{

  \renewcommand{\DataPath}{\FigurePath/#2/hp/order_1}%\PbName/hp
  \findmax{\DataPath/outputs.txt}{iter}{\endIter}
  %\findmin{\DataPath/outputs.txt}{iter}{\initIter} %\plothplevel

  \FormatIntegerThreeDigit{\endIter}{\endIterformated}
  \foreach \idim in {x,y,z}{
      \begin{subfigure}{\subplotwidth}
        \plothplevel{#2/hp/order_1}{\endIterformated}{\idim}
        \caption{Final $hp$-adapted mesh with polynomial orders in the $\idim$-direction.}
        \label{fig:#2_\idim}
      \end{subfigure}
    }

  %\pgfmathparse{subtract(\endIter,0)}
  %\pgfmathtruncatemacro\endIterIndex{\pgfmathresult}

  %\getelem{\DataPath/outputs.txt}{nr_dof}{\endIter}{\nrdofHpEnd}
  %\getelemdbl{\DataPath/outputs.txt}{Error_H1}{\endIter}{\errorHpEnd}
}

%%%%%%%%%%%%%%%%%%%%%%%%%% \subcaption{\captionifthenproblem{#2} problem. Evolution of the error in \captionifthenlegend{#2}.}
%: \captionifthenproblem
\newcommand{\captionifthenproblem}[1]{%
\ifthenelse{\equal{#1}{Regular1D}}{Regular solution}{%
\ifthenelse{\equal{#1}{CrossGOA}}{$e_{\textrm{rel}}^{\textrm{QoI}}$}{%
\ifthenelse{\equal{#1}{intBL}}{$e_{\textrm{rel}}^{\textrm{energy}}$}{%
\ifthenelse{\equal{#1}{intBL}}{boundary layer}{%
\ifthenelse{\equal{#1}{ConvDiff2DGOA}}{$e_{\textrm{rel}}^{\textrm{QoI}}$}{%
\ifthenelse{\equal{#1}{Helm2DGOA}}{$e_{\textrm{rel}}^{\textrm{QoI}}$}{%
\ifthenelse{\equal{#1}{Helm2DGOANon}}{$\tilde{e}_{\textrm{rel}}^{\textrm{\, energy}}$}{%
\ifthenelse{\equal{#1}{Helm3DGOA}}{$e_{\textrm{rel}}^{\textrm{QoI}}$}{%
\ifthenelse{\equal{#1}{Helm3DGOANon}}{$\tilde{e}_{\textrm{rel}}^{\textrm{\, energy}}$}{%
\ifthenelse{\equal{#1}{ConvDiff2DGOA-4}}{$e_{\textrm{rel}}^{\textrm{QoI}}$}{%
\ifthenelse{\equal{#1}{ConvDiff2DGOAtheta}}{$e_{\textrm{rel}}^{\textrm{QoI}}$}{%
\ifthenelse{\equal{#1}{ConvDiff2DGOANon}}{$\tilde{e}_{\textrm{rel}}^{\textrm{\, energy}}$}{%
\ifthenelse{\equal{#1}{atancircle2D}}{Shock}{prout!}%
}}}}}}}}}}}}%
}
%: \captionifthenlegend{#2}
\newcommand{\captionifthenlegend}[1]{%
  \ifthenelse{\equal{#1}{Regular1D}}{$H^1$ semi-norm}{%
    \ifthenelse{\equal{#1}{intBL}}{the norm $\norm{\cdot}_{\H}$}{%
      \ifthenelse{\equal{#1}{ConvDiff2DGOANon}}{$e_{\textrm{rel}}^{\textrm{QoI}}$}{%
        \ifthenelse{\equal{#1}{CrossGOA}}{the QoI}{%
          \ifthenelse{\equal{#1}{lap2DLshapeGOA}}{the QoI}{%
            \ifthenelse{\equal{#1}{ConvDiff2DGOA}}{the QoI}{%
              \ifthenelse{\equal{#1}{Helm2DGOA}}{the QoI}{%
                \ifthenelse{\equal{#1}{Helm2DLshapeGOA}}{the QoI}{%
                  \ifthenelse{\equal{#1}{Helm2DLshapeGOANon}}{the QoI}{%
                    \ifthenelse{\equal{#1}{Helm2DGOANon}}{$e_{\textrm{rel}}^{\textrm{QoI}}$}{%
                      \ifthenelse{\equal{#1}{Helm2DGOAalternative5}}{the QoI}{%
                        \ifthenelse{\equal{#1}{Helm2DGOAspd5}}{the QoI}{%
                          \ifthenelse{\equal{#1}{expBL}}{$H^1$ semi-norm}{%
                            \ifthenelse{\equal{#1}{Helm2D}}{the norm $\norm{\cdot}_{\H}$}{%
                              \ifthenelse{\equal{#1}{singularsquare2D}}{$H^1$ semi-norm}{%
                                \ifthenelse{\equal{#1}{ConvDiff2Dvcvd}}{$H^1$ semi-norm}{%
                                  \ifthenelse{\equal{#1}{singular2D}}{$H^1$ semi-norm}{%
                                    \ifthenelse{\equal{#1}{atan2D}}{$H^1$ semi-norm}{%
                                      \ifthenelse{\equal{#1}{atancircle2D}}{$H^1$ semi-norm}{prout!}%
                                    }}}}}}}}}}}}}}}}}}%
}

%: elemental routines

\newcommand{\FormatIntegerTwoDigit}[2]{
  \pgfmathsetbasenumberlength{2}
  \pgfmathbasetodec#2{#1}{10}
}

\newcommand{\FormatIntegerThreeDigit}[2]{ %Aquí INICIA la DEFINICION del COMANDO \FormatIntegerThreeDigit
  \pgfmathsetbasenumberlength{3}
  \pgfmathbasetodec#2{#1}{10}
}


\newcommand{\getelem}[4]{
  \pgfplotstablegetelem{#3}{#2}\of{#1}%
  \pgfmathtruncatemacro#4{\pgfplotsretval}
  \pgfplotstableclear{\datatable}
}


\newcommand{\getelemdbl}[4]{
  \pgfplotstablegetelem{#3}{#2}\of{#1}%
  \pgfmathsetmacro#4{\pgfplotsretval}
  \pgfplotstableclear{\datatable}
}

%   \pgfplotstableread{\datatable}{#1}
%   \pgfplotstablecreatecol[
%     create col/expr={%
%         \pgfplotstablerow
%       }]{rownumber}\datatable
%   \pgfplotstablesort[sort key={#2},sort cmp={float >}]{\sorted}{\datatable}%
%   \pgfplotstablegetelem{0}{#2}\of{\sorted}%
%   \pgfmathtruncatemacro#3{\pgfplotsretval}
%   \pgfplotstableclear{\datatable}
%   %
% }

% \newcommand{\findmin}[3]{
%   \pgfplotstablevertcat{\datatable}{#1}
%   \pgfplotstablecreatecol[
%     create col/expr={%
%         \pgfplotstablerow
%       }]{rownumber}\datatable
%   \pgfplotstablesort[sort key={#2},sort cmp={float <}]{\sorted}{\datatable}%
%   \pgfplotstablegetelem{0}{#2}\of{\sorted}%
%   \pgfmathtruncatemacro#3{\pgfplotsretval}
%   \pgfplotstableclear{\datatable}
%   %
% }

% \newcommand{\findmaxDbl}[3]{
%   \pgfplotstablevertcat{\datatable}{#1}
%   \pgfplotstablecreatecol[
%     create col/expr={%
%         \pgfplotstablerow
%       }]{rownumber}\datatable
%   \pgfplotstablesort[sort key={#2},sort cmp={float >}]{\sorted}{\datatable}%
%   \pgfplotstablegetelem{0}{#2}\of{\sorted}%
%   \pgfmathsetmacro#3{\pgfplotsretval}
%   \pgfplotstableclear{\datatable}
%   %
% }


\newcommand{\findmax}[3]{
  \pgfplotstableread{#1}{\datatable}
  % \pgfplotstablevertcat{\datatable}{#1}
  % \pgfplotstablecreatecol[
  % create col/expr={%
  % \pgfplotstablerow
  % }]{rownumber}\datatable
  \pgfplotstablesort[sort key={#2},sort cmp={float >}]{\sorted}{\datatable}%
  \pgfplotstablegetelem{0}{#2}\of{\sorted}%
  \pgfmathtruncatemacro#3{\pgfplotsretval}
  \pgfplotstableclear{\datatable}
  %
}

\newcommand{\findmin}[3]{
  \pgfplotstableread{#1}{\datatable}
  % \pgfplotstablevertcat{\datatable}{#1}
  % \pgfplotstablecreatecol[
  %   create col/expr={%
  % \pgfplotstablerow
  % }]{rownumber}\datatable
  \pgfplotstablesort[sort key={#2},sort cmp={float <}]{\sorted}{\datatable}%
  \pgfplotstablegetelem{0}{#2}\of{\sorted}%
  \pgfmathtruncatemacro#3{\pgfplotsretval}
  \pgfplotstableclear{\datatable}
  %
}

\newcommand{\findmaxDbl}[3]{
  % \pgfplotstablevertcat{\datatable}{#1}
  \pgfplotstableread{#1}{\datatable}
  % \pgfplotstablecreatecol[
  % create col/expr={%
  % \pgfplotstablerow
  % }]{rownumber}\datatable
  \pgfplotstablesort[sort key={#2},sort cmp={float >}]{\sorted}{\datatable}%
  \pgfplotstablegetelem{0}{#2}\of{\sorted}%
  \pgfmathsetmacro#3{\pgfplotsretval}
  \pgfplotstableclear{\datatable}
  %
}


\newcommand{\lastelement}[3]{
  \pgfplotstableread{#1}{\mytable}
  \pgfplotstablegetrowsof{\mytable} %Determine no. of rows
  \pgfmathtruncatemacro{\rows}{\pgfplotsretval}
  \pgfmathtruncatemacro{\lastrow}{\rows-1}
  \pgfplotstablegetelem{\lastrow}{#2}\of{\mytable}
  \pgfmathsetmacro#3{\pgfplotsretval}
}


\newcommand{\findminDbl}[3]{
  \pgfplotstablevertcat{\datatable}{#1}
  \pgfplotstablecreatecol[
    create col/expr={%
        \pgfplotstablerow
      }]{rownumber}\datatable
  \pgfplotstablesort[sort key={#2},sort cmp={float <}]{\sorted}{\datatable}%
  \pgfplotstablegetelem{0}{#2}\of{\sorted}%
  \pgfmathsetmacro#3{\pgfplotsretval}
  \pgfplotstableclear{\datatable}
  %
}


\pgfplotsset{
  log x ticks with fixed point/.style={
      xticklabel={
          \pgfkeys{/pgf/fpu=true}
          \pgfmathparse{exp(\tick)}%
          \pgfmathprintnumber[fixed relative, precision=3]{\pgfmathresult}
          \pgfkeys{/pgf/fpu=false}
        }
    },
  log y ticks with fixed point/.style={
      yticklabel={
          \pgfkeys{/pgf/fpu=true}
          \pgfmathparse{exp(\tick)}%
          \pgfmathprintnumber[fixed relative, precision=3]{\pgfmathresult}
          \pgfkeys{/pgf/fpu=false}
        }
    }
}



\pgfplotscreateplotcyclelist{custom list style}{%
  color=red, solid, every mark/.append style={solid}, mark=*\\%
  color=blue, loosely dotted, every mark/.append style={solid}, mark=triangle*\\%
  color=black!50!green,dotted, every mark/.append style={solid}, mark=square*\\%
  color=black, dashed, every mark/.append style={solid},mark=diamond*\\%
  color=black!50!red, densely dotted, every mark/.append style={solid}, mark=otimes*\\%
  color=black!50!yellow, loosely dashed, every mark/.append style={solid},mark=*\\%
  color=black!50!blue,densely dashed, every mark/.append style={solid},mark=square*\\%
  color=black!50!magenta,dashdotted, every mark/.append style={solid},mark=otimes*\\%
  color=black!50!orange,dashdotdotted, every mark/.append style={solid},mark=star\\%
  color=black!50!cyan, densely dashdotted,every mark/.append style={solid},mark=diamond*\\%
}

\pgfplotsset{%
  every axis plot/.append style= {line width=2pt},
  cycle list name=custom list style,
  %    log x ticks with fixed point,
  %    legend style={at={(0.5,1.03)},anchor=south},
  %    legend columns=3,
  %    legend pos= outer north east,
  %    legend cell align=left,
}
